%%%%%%%%%%%%%%%%%%%%%%%%%%%%%%%%%%%%%%%%%
% The Legrand Orange Book
% LaTeX Template
% Version 2.4 (26/09/2018)
%
% This template was downloaded from:
% http://www.LaTeXTemplates.com
%
% Original author:
% Mathias Legrand (legrand.mathias@gmail.com) with modifications by:
% Vel (vel@latextemplates.com)
%
% License:
% CC BY-NC-SA 3.0 (http://creativecommons.org/licenses/by-nc-sa/3.0/)
%
% Compiling this template:
% This template uses biber for its bibliography and makeindex for its index.
% When you first open the template, compile it from the command line with the
% commands below to make sure your LaTeX distribution is configured correctly:
%
% 1) pdflatex main
% 2) makeindex main.idx -s StyleInd.ist
% 3) biber main
% 4) pdflatex main x 2
%
% After this, when you wish to update the bibliography/index use the appropriate
% command above and make sure to compile with pdflatex several times
% afterwards to propagate your changes to the document.
%
% This template also uses a number of packages which may need to be
% updated to the newest versions for the template to compile. It is strongly
% recommended you update your LaTeX distribution if you have any
% compilation errors.
%
% Important note:
% Chapter heading images should have a 2:1 width:height ratio,
% e.g. 920px width and 460px height.
%
%%%%%%%%%%%%%%%%%%%%%%%%%%%%%%%%%%%%%%%%%

%----------------------------------------------------------------------------------------
%	PACKAGES AND OTHER DOCUMENT CONFIGURATIONS
%----------------------------------------------------------------------------------------

\documentclass[11pt,fleqn,openany]{book} % Default font size and left-justified equations
\input{structure.tex} % Insert the commands.tex file which contains the majority of the structure behind the template

%\hypersetup{pdftitle={Title},pdfauthor={Author}} % Uncomment and fill out to include PDF metadata for the author and title of the book

\usepackage{listings}
\usepackage{caption}
%----------------------------------------------------------------------------------------


\definecolor{mygray}{rgb}{0.5,0.5,0.5}
\lstdefinelanguage{jac}
{
    % list of keywords
    keywords={ node, ignore, take, entry, activity, exit, spawn, with,
            edge, walker, and, or, if, elif, else, for, with, by, while,
            continue, break, disengage, report, anchor, has, can, true, false
        },
    sensitive=false, % keywords are not case-sensitive
    morecomment=[l]{//}, % l is for line comment
    morecomment=[l]{\#}, % l is for line comment
    morecomment=[s]{/*}{*/}, % s is for start and end delimiter
    morestring=[b]", % defines that strings are enclosed in double quotes
    morestring=[b]',
}
\lstdefinelanguage{shell}
{
    % list of keywords
    keywords={
        },
}

\lstdefinestyle{gram}{
basicstyle=\footnotesize\ttfamily\color{olive},%
breaklines=true,%                                      allow line breaks
moredelim=[s][\color{green!50!black}\ttfamily]{'}{'},% single quotes in green
moredelim=*[s][\color{black}\ttfamily]{options}{\}},%  options in black (until trailing })
commentstyle={\color{gray}\itshape},%                  gray italics for comments
morecomment=[l]{//},%                                  define // comment
emph={%
        STRING%                                            literal strings listed here
    },emphstyle={\color{blue}\ttfamily},%              and formatted in blue
alsoletter={:,|,;},%
morekeywords={:,|,;},%                                 define the special characters
keywordstyle={\color{black}},%                         and format them in black
}
\renewcommand{\lstlistingname}{Jac Code}% Listing -> Jac Code
\renewcommand{\lstlistlistingname}{List of \lstlistingname s}% List of Listings -> List of Algorithms

% \DeclareCaptionFormat{listing}{#1#2#3}

% \captionsetup[lstlisting]{format=listing,singlelinecheck=false,margin=0pt,labelfont=bf}
\DeclareCaptionFont{white}{\color{white}\footnotesize\upshape\ttfamily}
\DeclareCaptionFormat{listing}{\colorbox{gray}{\parbox{\linewidth}{#1#2#3}}}
\captionsetup[lstlisting]{format=listing,labelfont=white,textfont=white,margin=0pt}
\lstset{ %
    language=jac,                                     % choose the language of the code
    frame=single,	                                % adds a frame around the code
    captionpos=t,
    basicstyle=\footnotesize\upshape\ttfamily,      % the size of the fonts that are used for the code
    numbers=none,                    % where to put the line-numbers; possible values are (none, left, right)
    numbersep=10pt,                   % how far the line-numbers are from the code
    numberstyle=\tiny\color{mygray}, % the style that is used for the line-numbers
    keywordstyle=\bfseries\color{green!55!black},
    commentstyle=\itshape\color{purple!70!black},
    identifierstyle=\color{blue!60!black},
    stringstyle=\color{orange},
    xleftmargin=3.5pt,
    xrightmargin=-2.5pt,
}



\begin{document}



% \lstset{
%   backgroundcolor=\color{white},   % choose the background color; you must add \usepackage{color} or \usepackage{xcolor}; should come as last argument
%   basicstyle=\footnotesize,        % the size of the fonts that are used for the code
%   breakatwhitespace=false,         % sets if automatic breaks should only happen at whitespace
%   breaklines=true,                 % sets automatic line breaking
%   captionpos=b,                    % sets the caption-position to bottom
%   commentstyle=\color{mygreen},    % comment style
%   deletekeywords={...},            % if you want to delete keywords from the given language
%   escapeinside={\%*}{*)},          % if you want to add LaTeX within your code
%   extendedchars=true,              % lets you use non-ASCII characters; for 8-bits encodings only, does not work with UTF-8
%   firstnumber=1000,                % start line enumeration with line 1000
%   frame=single,	                   % adds a frame around the code
%   keepspaces=true,                 % keeps spaces in text, useful for keeping indentation of code (possibly needs columns=flexible)
%   keywordstyle=\color{blue},       % keyword style
%   language=Octave,                 % the language of the code
%   morekeywords={*,...},            % if you want to add more keywords to the set
%   numbers=left,                    % where to put the line-numbers; possible values are (none, left, right)
%   numbersep=5pt,                   % how far the line-numbers are from the code
%   numberstyle=\tiny\color{mygray}, % the style that is used for the line-numbers
%   rulecolor=\color{black},         % if not set, the frame-color may be changed on line-breaks within not-black text (e.g. comments (green here))
%   showspaces=false,                % show spaces everywhere adding particular underscores; it overrides 'showstringspaces'
%   showstringspaces=false,          % underline spaces within strings only
%   showtabs=false,                  % show tabs within strings adding particular underscores
%   stepnumber=2,                    % the step between two line-numbers. If it's 1, each line will be numbered
%   stringstyle=\color{mymauve},     % string literal style
%   tabsize=2,	                   % sets default tabsize to 2 spaces
%   title=\lstname                   % show the filename of files included with \lstinputlisting; also try caption instead of title
% }
%----------------------------------------------------------------------------------------
%	TITLE PAGE
%----------------------------------------------------------------------------------------

\begingroup
\thispagestyle{empty} % Suppress headers and footers on the title page
\begin{tikzpicture}[remember picture,overlay]
    \node[inner sep=0pt] (background) at (current page.center) {\includegraphics[width=\paperwidth]{background.pdf}};
    \draw (current page.center) node [fill=ocre!30!white,fill opacity=0.6,text opacity=1,inner sep=1cm]
    {\Huge\centering\bfseries\sffamily\parbox[c][][t]{\paperwidth}
    {\centering The Jaseci Machine \& \\Jac Language\\[15pt] % Book title
    {\Large Processing Architecture and Programming Language for High Complexity AI}\\[20pt] % Subtitle
    {\huge Dr. Jason Mars}}}; % Author name
\end{tikzpicture}
\vfill
\endgroup

%----------------------------------------------------------------------------------------
%	COPYRIGHT PAGE
%----------------------------------------------------------------------------------------

\newpage
~\vfill
\thispagestyle{empty}

\noindent Copyright \copyright\ 2020 Dr. Jason Mars\\ % Copyright notice

\noindent \textsc{Published by Publisher}\\ % Publisher

\noindent \textsc{jasonmars.org}\\ % URL

% \noindent Licensed under the Creative Commons Attribution-NonCommercial 3.0 Unported License (the ``License''). You may not use this file except in compliance with the License. You may obtain a copy of the License at \url{http://creativecommons.org/licenses/by-nc/3.0}. Unless required by applicable law or agreed to in writing, software distributed under the License is distributed on an \textsc{``as is'' basis, without warranties or conditions of any kind}, either express or implied. See the License for the specific language governing permissions and limitations under the License.\\ % License information, replace this with your own license (if any)

\noindent \textit{First printing, September 2020} % Printing/edition date

%----------------------------------------------------------------------------------------
%	TABLE OF CONTENTS
%----------------------------------------------------------------------------------------

%\usechapterimagefalse % If you don't want to include a chapter image, use this to toggle images off - it can be enabled later with \usechapterimagetrue

\chapterimage{chapter_head.pdf} % Table of contents heading image

\pagestyle{empty} % Disable headers and footers for the following pages

\tableofcontents % Print the table of contents itself

% \cleardoublepage % Forces the first chapter to start on an odd page so it's on the right side of the book

\pagestyle{fancy} % Enable headers and footers again


%----------------------------------------------------------------------------------------
%   Main Content
%----------------------------------------------------------------------------------------
\chapterimage{chapter_head.pdf}\chapter{Introduction}\input{tex/chap/chap-intro}
\part{Jaseci Machine}\chapterimage{chapter_head.pdf}\chapter{Contexts and Actions}\section{Contexts}\index{context}In Jaseci, a \emph{context element} represents a data element with a corresponding unique identifier of that data element. This abstraction is analogous to the traditional view of addressible memory in a computer system [CITE] for which which each word (data value) is addressible with a 32-bit or 64-bit memory address (identifier). However, a context element is untyped and in principle unbounded in size. Context have no requirements or restrictions for the kind or type of data element that is stored and only requires the identifier to be unique.

Though contexts are sufficent to enable all representations and management of data in the Jaseci machine, we also introduce the abstraction of \emph{context sets} that represents an explicit set of contexts bound to a single identifier. This \emph{context set} abstraction provides the utility of grouping, organizing, and handeling collections of related contexts as a single unit.
\begin{remark}
    \begin{tBox}
        The practical implmenetation of a Jaseci machine described in this book uses URN UUIDs [CITE] for the identifer.
    \end{tBox}
\end{remark}

\subsection{Formal Definition of Contexts and Context Sets}

Definitions~\ref{def:context} and~\ref{def:context_set} formally describe contexts and context sets.

\begin{definition}[context]
    \label{def:context}
    \index{context}
    A \emph{context} is a representation of data that can be expressed as a $2$-tuple $(k,v)$, where
    \begin{enumerate}
        \item $k$ is a unique identifier of the context
        \item $v$ is an encoding of the data represented by the context
    \end{enumerate}
\end{definition}

\begin{definition}[context set]
    \label{def:context_set}
    \index{context set}
    A \emph{context set} is a representation of data that can be expressed as a $2$-tuple $(k,C)$, where
    \begin{enumerate}
        \item $k$ is a unique identifier of the context set
        \item $C$ is an explicit finite set of contexts and context sets
    \end{enumerate}
\end{definition}



\subsection{Example}
\begin{example}
    For each trip a shopper has made to a grocery store, suppose we'd like to use contexts to represent what the shopper bought and the most expensive item.
    Let the set $I = \{ item_{1},\:item_{2},\:item_{3},\:\dots,\:item_{n}\}$ be the set of all distinct items in a given store and $P = \{item | item \in I \:\text{and was paid for by shopper}\}$ . We construct the context and context set
    \begin{align}
        A & = (k,\:item_{i}\:|\:\text{$item_{i}$ is the first $item \in I \land P$ sorted by cost.}) \\
        B & = (k, I \land P)
    \end{align}
\end{example}
% \section{Actions}\index{action}\subsection{Formal Definition of Contexts and Context Sets}

Definitions~\ref{def:context} formally describe contexts and context sets.

\begin{definition}[context]\index{Contexts}
    \label{def:action}
    A \emph{context} is a representation of data that can be expressed as a $2$-tuple $(k,v)$, where
    \begin{enumerate}
        \item $k$ is a unique identifier of the context
        \item $v$ is an encoding of the data represented by the context
    \end{enumerate}
\end{definition}



\chapterimage{chapter_head.pdf}\chapter{Jaseci Graphs, Nodes and Edges}
\section{Graphs}\index{node}\input{tex/sec/sec-graphs}
\section{Nodes}\index{node}\input{tex/sec/sec-nodes}
\section{Edges}\index{edge}\input{tex/sec/sec-edges}

\chapterimage{chapter_head.pdf}\chapter{Multidimensional Graph Planes and Domains}\section{Graph Planes}\index{graph plane}\input{tex/sec/sec-planes}
\section{Domain Nodes}\index{domain node}\input{tex/sec/sec-domains}

\chapterimage{chapter_head.pdf}\chapter{Walkers, Architypes and Sentinels}
\section{Architypes}\index{domain node}
\subsection{Formal Definition}

\begin{description}
    \item[Name] Graph \vspace{2mm}
    \item[Word] \begin{definition}[action]
              A \textbf{\index{action}} is a representation of data that can be expressed as a $2$-tuple $(k,v)$, where
              \begin{enumerate}
                  \item $k$ is a unique identifier of the context
                  \item $v$ is an encoding of the data represented by the context
              \end{enumerate}
          \end{definition}
    \item[Description] Go deeper
\end{description}

\section{Walker}\index{walker}\input{tex/sec/sec-walkers}
\section{Sentinels}\index{sentinel}\input{tex/sec/sec-sentinels}


\chapterimage{chapter_head.pdf}\chapter{A Jaseci Machine}\section{A Jaseci Machine}\index{sentinel}\input{tex/sec/sec-machine}

\part{Jac Language}\chapterimage{chapter_head.pdf}\chapter{Language Overview and Basics}Jac is dynamically typed.

Jac is pass by reference.

Jac is whitespace agnostic.

Jac does not have traditional notion of functions but walkers instead. Deliberate to encourage thought about problems in a spacial graph paradigmn that is equivilant in power but more natural for many complex AI problems.

\begin{lstlisting}[caption={Hello World},label={lst:hello}]
walker init {
    std.log('Hello World');
}
\end{lstlisting}

Listing~\ref{lst:hello} shows the canonical 'hello world' program as it would be most simply implemented in Jac.

Jac also has the notion of single statement line code blocks.

\begin{lstlisting}[caption={Hello World},label={lst:hellosingle}]
walker init: std.log('Hello World');
\end{lstlisting}


Program~\ref{lst:hello} is equivilant to Program~\ref{lst:hellosingle}.
% Chapter \ref{} describes walkers and Chapter~\ref{} describes Jac's standard library of Actions.


\section{Numbers and Arithmetic}\index{numbers in Jac}%%%%%%%%%%%%%%%%%%%%%%%%%%%%%%
\subsection*{Basic Arithmetic Operations}
The simplest math operations in Jac.
\begin{description}
    \item[Code] \texttt{}
          \begin{lstlisting}[caption={Basic arithmetic operations}]
walker init {
    a = 4 + 4;
    b = 4 * -5;
    c = 4 / 4;  # Evaluates to a floating point number
    d = 4 - 6;
    e = a + b + c + d;
    std.out(a, b, c, d, e);
}
    \end{lstlisting}
    \item[Output] \texttt{ }
          \begin{lstlisting}[language=shell]
8 -20 1.0 -2 -13.0
        \end{lstlisting}
    \item[Description] \texttt{}
\end{description}

\noindent Additionally, Jac supports power and modulo operations.
\begin{description}
    \begin{lstlisting}[caption={Additional arithmetic operations}]
walker init {
    a = 4 ^ 4; b = 9 % 5; std.out(a, b);
}
    \end{lstlisting}
    \item[Output] \texttt{ }
          \begin{lstlisting}[language=shell]
256 4
        \end{lstlisting}
    \item[Description] \texttt{}
\end{description}


%%%%%%%%%%%%%%%%%%%%%%%%%%%%%%
\subsection*{Comparison Operations}
\begin{description}
    \begin{lstlisting}[caption={Comparision operations}]
walker init {
    a = 5; b = 6;
    std.out(a == b,
            a != b,
            a < b,
            a > b,
            a <= b,
            a >= b,
            a == b-1);
}
    \end{lstlisting}
    \item[Output] \texttt{ }
          \begin{lstlisting}[language=shell]
false true true false true false true
        \end{lstlisting}
    \item[Description] \texttt{}
\end{description}


%%%%%%%%%%%%%%%%%%%%%%%%%%%%%%
\subsection*{Logical Operations}
\begin{description}
    \begin{lstlisting}[caption={Logical operations}]
walker init {
    a = true; b = false;
    std.out(a,
            !a,
            a && b,
            a || b,
            a and b,
            a or b,
            !a or b,
            !(a and b));
}
    \end{lstlisting}
    \item[Output] \texttt{ }
          \begin{lstlisting}[language=shell]
true false false true false true false true
        \end{lstlisting}
    \item[Description] \texttt{}
\end{description}


%%%%%%%%%%%%%%%%%%%%%%%%%%%%%%
\subsection*{Assignment Operations}
\begin{description}
    \begin{lstlisting}[caption={Assignment operations}]
walker init {
    a = 4 + 4; std.out(a);
    a += 4 + 4; std.out(a);
    a -= 4 * -5; std.out(a);
    a *= 4 / 4; std.out(a);
    a /= 4 - 6; std.out(a);

    # a := here; std.out(a);
    # Noting existence of copy assign, described later
}
    \end{lstlisting}
    \item[Output] \texttt{ }
          \begin{lstlisting}[language=shell]
8
16
36
36.0
-18.0
        \end{lstlisting}
    \item[Description] \texttt{}
\end{description}


%%%%%%%%%%%%%%%%%%%%%%%%%%%%%%
\subsection*{Foreshadowing Unique Graph Operations}
\begin{description}
    \begin{lstlisting}[caption={Preview of graph operators},
        label={code:moremath}]
edge back;

walker init {
    node_a = spawn node;
    here --> node_a;
    here <-[back]- node_a;

    node_b = spawn here <-> node;
    node_b --> node_a
}
    \end{lstlisting}
    \item[Output] \texttt{ }
    \item[Description] \texttt{}

          \begin{tikzpicture}[node distance = {1.0cm and 1.5cm}, v/.style = {draw, circle}]
              \graph[nodes={circle, draw}, grow right=2.25cm, branch down=1.75cm]{
              H -> A,
              A -> ["back"] H,
              H -- B,
              B -> A,
              };
          \end{tikzpicture}
\end{description}


%%%%%%%%%%%%%%%%%%%%%%%%%%%%%%
\subsection*{Precedence}

\begin{table}[h]
    \small
    \centering
    \begin{tabular}{l l l}
        \toprule
        \textbf{Rank} & \textbf{Symbol}          & \textbf{Description}                           \\
        \midrule
        1             & () [] . :: --> <-- spawn & Parenthetical/grouping, node/edge manipulation \\
        2             & \textasciicircum         & Exponent                                       \\
        3             & * / \%                   & Multiplication, division, modulo               \\
        4             & + -                      & Addition, subtraction                          \\
        5             & == != >= <= > <          & Comparison                                     \\
        6             & \&\& || and or           & Logical                                        \\
        7             & = += -= *= /= :=         & Assignment                                     \\
        \bottomrule
    \end{tabular}
    \caption{Precedence of operations in Jac}
    \label{tab:jacprecedence} % Unique label used for referencing the table in-text
    %\addcontentsline{toc}{table}{Table \ref{tab:jacprecedence}} % Uncomment to add the table to the table of contents
\end{table}
\section{Strings and List}\index{strings in Jac}\index{lists in Jac} Coming soon.%%%%%%%%%%%%%%%%%%%%%%%%%%%%%%%
\subsection*{Basic Arithmetic Operations}
The simplest math operations in Jac.
\begin{description}
    \item[Code] \texttt{}
          \begin{lstlisting}[caption={Basic arithmetic operations}]
walker init {
    a = 4 + 4;
    b = 4 * -5;
    c = 4 / 4;  # Evaluates to a floating point number
    d = 4 - 6;
    e = a + b + c + d;
    std.out(a, b, c, d, e);
}
    \end{lstlisting}
    \item[Output] \texttt{ }
          \begin{lstlisting}[language=shell]
8 -20 1.0 -2 -13.0
        \end{lstlisting}
    \item[Description] \texttt{}
\end{description}

\noindent Additionally, Jac supports power and modulo operations.
\begin{description}
    \begin{lstlisting}[caption={Additional arithmetic operations}]
walker init {
    a = 4 ^ 4; b = 9 % 5; std.out(a, b);
}
    \end{lstlisting}
    \item[Output] \texttt{ }
          \begin{lstlisting}[language=shell]
256 4
        \end{lstlisting}
    \item[Description] \texttt{}
\end{description}


%%%%%%%%%%%%%%%%%%%%%%%%%%%%%%
\subsection*{Comparison Operations}
\begin{description}
    \begin{lstlisting}[caption={Comparision operations}]
walker init {
    a = 5; b = 6;
    std.out(a == b,
            a != b,
            a < b,
            a > b,
            a <= b,
            a >= b,
            a == b-1);
}
    \end{lstlisting}
    \item[Output] \texttt{ }
          \begin{lstlisting}[language=shell]
false true true false true false true
        \end{lstlisting}
    \item[Description] \texttt{}
\end{description}


%%%%%%%%%%%%%%%%%%%%%%%%%%%%%%
\subsection*{Logical Operations}
\begin{description}
    \begin{lstlisting}[caption={Logical operations}]
walker init {
    a = true; b = false;
    std.out(a,
            !a,
            a && b,
            a || b,
            a and b,
            a or b,
            !a or b,
            !(a and b));
}
    \end{lstlisting}
    \item[Output] \texttt{ }
          \begin{lstlisting}[language=shell]
true false false true false true false true
        \end{lstlisting}
    \item[Description] \texttt{}
\end{description}


%%%%%%%%%%%%%%%%%%%%%%%%%%%%%%
\subsection*{Assignment Operations}
\begin{description}
    \begin{lstlisting}[caption={Assignment operations}]
walker init {
    a = 4 + 4; std.out(a);
    a += 4 + 4; std.out(a);
    a -= 4 * -5; std.out(a);
    a *= 4 / 4; std.out(a);
    a /= 4 - 6; std.out(a);

    # a := here; std.out(a);
    # Noting existence of copy assign, described later
}
    \end{lstlisting}
    \item[Output] \texttt{ }
          \begin{lstlisting}[language=shell]
8
16
36
36.0
-18.0
        \end{lstlisting}
    \item[Description] \texttt{}
\end{description}


%%%%%%%%%%%%%%%%%%%%%%%%%%%%%%
\subsection*{Foreshadowing Unique Graph Operations}
\begin{description}
    \begin{lstlisting}[caption={Preview of graph operators},
        label={code:moremath}]
edge back;

walker init {
    node_a = spawn node;
    here --> node_a;
    here <-[back]- node_a;

    node_b = spawn here <-> node;
    node_b --> node_a
}
    \end{lstlisting}
    \item[Output] \texttt{ }
    \item[Description] \texttt{}

          \begin{tikzpicture}[node distance = {1.0cm and 1.5cm}, v/.style = {draw, circle}]
              \graph[nodes={circle, draw}, grow right=2.25cm, branch down=1.75cm]{
              H -> A,
              A -> ["back"] H,
              H -- B,
              B -> A,
              };
          \end{tikzpicture}
\end{description}


%%%%%%%%%%%%%%%%%%%%%%%%%%%%%%
\subsection*{Precedence}

\begin{table}[h]
    \small
    \centering
    \begin{tabular}{l l l}
        \toprule
        \textbf{Rank} & \textbf{Symbol}          & \textbf{Description}                           \\
        \midrule
        1             & () [] . :: --> <-- spawn & Parenthetical/grouping, node/edge manipulation \\
        2             & \textasciicircum         & Exponent                                       \\
        3             & * / \%                   & Multiplication, division, modulo               \\
        4             & + -                      & Addition, subtraction                          \\
        5             & == != >= <= > <          & Comparison                                     \\
        6             & \&\& || and or           & Logical                                        \\
        7             & = += -= *= /= :=         & Assignment                                     \\
        \bottomrule
    \end{tabular}
    \caption{Precedence of operations in Jac}
    \label{tab:jacprecedence} % Unique label used for referencing the table in-text
    %\addcontentsline{toc}{table}{Table \ref{tab:jacprecedence}} % Uncomment to add the table to the table of contents
\end{table}
\section{Control Flow}\index{control in Jac}%%%%%%%%%%%%%%%%%%%%%%%%%%%%%%
\subsection*{Conditional Statements}
\begin{description}
    \begin{lstlisting}[caption={\texttt{if} statements}]
walker init {
    a = 4; b = 5;
    if(a < b): std.out("Hello!");
}
    \end{lstlisting}
    \item[Output] \texttt{}
          \begin{lstlisting}[language=shell]
Hello!
        \end{lstlisting}
    \item[Description] \texttt{}
\end{description}

\begin{description}
    \begin{lstlisting}[caption={\texttt{else} statement}]
walker init {
    a = 4; b = 5;
    if(a == b): std.out("A equals B");
    else: std.out("A is not equal to B");
}
    \end{lstlisting}
    \item[Output] \texttt{}
          \begin{lstlisting}[language=shell]
A is not equal to B
        \end{lstlisting}
    \item[Description] \texttt{}
\end{description}

\begin{description}
    \begin{lstlisting}[caption={\texttt{elif} statement}]
walker init {
    a = 4; b = 5;
    if(a == b): std.out("A equals B");
    elif(a > b): std.out("A is greater than B");
    elif(a == b - 1): std.out("A is one less than B");
    elif(a == b - 2): std.out("A is two less than B");
    else: std.out("A is something else");
}
    \end{lstlisting}
    \item[Output] \texttt{}
          \begin{lstlisting}[language=shell]
A is one less than B
        \end{lstlisting}
    \item[Description] \texttt{}
\end{description}


%%%%%%%%%%%%%%%%%%%%%%%%%%%%%%
\subsection*{Loops}
\begin{description}
    \begin{lstlisting}[caption={\texttt{for} loops}]
walker init {
    for i=0 to i<10 by i+=1:
        std.out("Hello", i, "times!");
}
    \end{lstlisting}
    \item[Output] \texttt{}
          \begin{lstlisting}[language=shell]
Hello 0 times!
Hello 1 times!
Hello 2 times!
Hello 3 times!
Hello 4 times!
Hello 5 times!
Hello 6 times!
Hello 7 times!
Hello 8 times!
Hello 9 times!
        \end{lstlisting}
    \item[Description] \texttt{}
\end{description}

\begin{description}
    \begin{lstlisting}[caption={\texttt{for} loops iterating through lists}]
walker init {
    my_list = [1, 'jon', 3.5, 4];
    for i in my_list:
        std.out("Hello", i, "times!");
}
    \end{lstlisting}
    \item[Output] \texttt{}
          \begin{lstlisting}[language=shell]
TEST CASE NOT GENERATED YET
        \end{lstlisting}
    \item[Description] \texttt{}



          \begin{remark}
              \begin{tBox}
                  Remember, though this looks very much like python, the \texttt{:} operator here indicates single line block. Braces should be used for multiline code blocks, e.g.,
                  \begin{lstlisting}
for i in my_list {
    if(i == 'jon'): i = 5;
    std.out("Hello", i, "times!");
}
        \end{lstlisting}
              \end{tBox}
          \end{remark}
\end{description}

\begin{description}
    \begin{lstlisting}[caption={ \texttt{while} loops }]
walker init {
    i = 5;
    while(i>0) {
        std.out("Hello", i, "times!");
        i -= 1;
    }
}
    \end{lstlisting}
    \item[Output] \texttt{}
          \begin{lstlisting}[language=shell]
Hello 5 times!
Hello 4 times!
Hello 3 times!
Hello 2 times!
Hello 1 times!
        \end{lstlisting}
    \item[Description] \texttt{}
\end{description}


%%%%%%%%%%%%%%%%%%%%%%%%%%%%%%
\subsection*{Loop Control Statements}
\begin{description}
    \begin{lstlisting}[caption={\texttt{break} statement}]
walker init {
    for i=0 to i<10 by i+=1 {
        std.out("Hello", i, "times!");
        if(i == 6): break;
    }
}
    \end{lstlisting}
    \item[Output] \texttt{}
          \begin{lstlisting}[language=shell]
Hello 0 times!
Hello 1 times!
Hello 2 times!
Hello 3 times!
Hello 4 times!
Hello 5 times!
Hello 6 times!
        \end{lstlisting}
    \item[Description] \texttt{}
\end{description}

\begin{description}
    \begin{lstlisting}[caption={\texttt{continue} statement}]
walker init {
    i = 5;
    while(i>0) {
        if(i == 3){
            i -= 1; continue;
        }
        std.out("Hello", i, "times!");
        i -= 1;
    }
}
    \end{lstlisting}
    \item[Output] \texttt{}
          \begin{lstlisting}[language=shell]
Hello 5 times!
Hello 4 times!
Hello 2 times!
Hello 1 times!
        \end{lstlisting}
    \item[Description] \texttt{}
\end{description}
\chapterimage{chapter_head.pdf}\chapter{Archetypes and Walkers}
Define Architiype

Define Walkers

Describe their interaction and roles


\section{Walkers and Graphs as First Order Citizens}\index{walkers in Jac}\index{graphs in Jac}Define and describe walker in Jac

Introduce graphs, nodes, edges


\begin{description}
    \begin{lstlisting}[caption={Simple walker creating and connected nodes}]
walker init {
    node1 = spawn node;
    node2 = spawn node;
    node1 <-> node2;
    here --> node1;
    node2 <-- here;
}
    \end{lstlisting}
    \item[Output] \texttt{}
          \begin{lstlisting}[language=shell]
        \end{lstlisting}
    \item[Description] \texttt{}
\end{description}


\begin{description}
    \begin{lstlisting}[caption={Creating named node types}]
node person;
edge assists;
edge family;

walker init {
    node1 = spawn node::person;
    node2 = spawn node::person;
    node1 <-[family]-> node2;
    here -[friend]-> node1;
    node2 <-[friend]- here;

    # named and unnamed edges and nodes can be mixed
    node2 --> here;
}
    \end{lstlisting}
    \item[Output] \texttt{}
          \begin{lstlisting}[language=shell]
        \end{lstlisting}
    \item[Description] \texttt{}
\end{description}


\begin{description}
    \begin{lstlisting}[caption={Connecting nodes within \texttt{spawn} statement}]
node person;
edge assists;
edge family;

walker init {
    node1 = spawn here -[friend]-> node::person;
    node2 = spawn node1 <-[family]-> node::person;
    here -[friend]-> node2;
}
    \end{lstlisting}
    \item[Output] \texttt{}
          \begin{lstlisting}[language=shell]
        \end{lstlisting}
    \item[Description] \texttt{}
\end{description}


\begin{description}
    \begin{lstlisting}[caption={Chaining node connections using the connect operator}]
node person;
edge assists;
edge family;

walker init {
    node1 = spawn node::person;
    node2 = spawn node::person;
    node2 <-[friend]- here -[friend]-> node1
          <-[family]-> node2;
}
    \end{lstlisting}
    \item[Output] \texttt{}
          \begin{lstlisting}[language=shell]
        \end{lstlisting}
    \item[Description] \texttt{}
\end{description}

\begin{description}
    \begin{lstlisting}[caption={Walkers spawning other walkers}]
node person;
edge assists;
edge family;

walker family_ties {
    for i in -[family]->:
        std.out(here, ' is related to ', i);
}

walker init {
    node1 = spawn here -[friend]-> node::person;
    node2 = spawn node1 <-[family]-> node::person;
    here -[friend]-> node2;
    spawn here walker::family_ties;
}
    \end{lstlisting}
    \item[Output] \texttt{}
          \begin{lstlisting}[language=shell]
        \end{lstlisting}
    \item[Description] \texttt{}
\end{description}



\begin{description}
    \begin{lstlisting}[caption={Getting returned values from spawned walkers}]
node person;
edge assists;
edge family;

walker family_ties {
    has anchor fam_nodes;
    fam_nodes = -[family]->:
}

walker init {
    node1 = spawn here -[friend]-> node::person;
    node2 = spawn node1 <-[family]-> node::person;
    here -[friend]-> node2;
    fam = spawn here walker::family_ties;
    for i in fam:
        std.out(here, ' is related to ', i);
}
    \end{lstlisting}
    \item[Output] \texttt{}
          \begin{lstlisting}[language=shell]
        \end{lstlisting}
    \item[Description] \texttt{}
          \begin{remark}
              \begin{tBox}
                  Remember \texttt{spawn} statements are expressions so they can be used as such, e.g.,
                  \begin{lstlisting}
for i in spawn here walker::family_ties:
    std.out(here, ' is related to ', i);
        \end{lstlisting}
              \end{tBox}
          \end{remark}
\end{description}


\section{Architypes and Actions}\index{architypes in Jac}\index{actions in Jac}
Nodes, and edges

Spawning nodes

Binding compute to architypes



\begin{description}
    \begin{lstlisting}[caption={Binding member contexts to nodes and edges}]
node person {
    has name;
    has age;
    has birthday, profession;
}

edge friend: has meeting_place;
edge family: has type;

walker init {
    person1 = spawn here -[friend]-> node::person;
    person2 = spawn here -[family]-> node::person;
    person1.name = "Josh"; person1.age = 32;
    person2.name = "Jane"; person2.age = 30;
    -[friend]->[0].meeting_place = "college";
    -[family]->[0].type = "sister"
    std.out(--> node);
}
    \end{lstlisting}
    \item[Output] \texttt{}
          \begin{lstlisting}[language=shell]
        \end{lstlisting}
    \item[Description] \texttt{}
\end{description}


\begin{description}
    \begin{lstlisting}[caption={Binding contexts with less code}]
node person {
    has name;
    has age;
    has birthday, profession;
}

edge friend: has meeting_place;
edge family: has type;

walker init {
    person1 = spawn here -[friend(meeting_place="college")]->
        node::person(name="Josh");
    person2 = spawn here -[family(type="sister")]->
        node::person(name="Jane");
    std.out(--> node);
}
    \end{lstlisting}
    \item[Output] \texttt{}
          \begin{lstlisting}[language=shell]
        \end{lstlisting}
    \item[Description] \texttt{}
\end{description}


\begin{description}
    \begin{lstlisting}[caption={Adding actions to architypes},label=jac:addactions]
node person {
    has name;
    has birthday;
    can date.quantize_to_year;
}

walker init {
    person1 = spawn here -->
        node::person(name="Josh", birthday="1995-05-20");
    birthyear = date.quantize_to_year(person1.birthday);
    std.out(birthyear);
}
    \end{lstlisting}
    \item[Output] \texttt{}
          \begin{lstlisting}[language=shell]
        \end{lstlisting}
    \item[Description] \texttt{}

\end{description}


\begin{description}
    \begin{lstlisting}[caption={Triggering actions on entry and exit}]
node person {
    has name;
    has bday, byear;
    can date.quantize_to_year::bday::>byear with entry;
    can std.out::byear," from ",bday:: with exit;
}

walker init {
    person1 = spawn here -->
        node::person(name="Josh", birthday="1995-05-20");
    take --> ;
    person: disengage;
}
    \end{lstlisting}
    \item[Output] \texttt{}
          \begin{lstlisting}[language=shell]
        \end{lstlisting}
    \item[Description] \texttt{}

          \begin{remark}
              \begin{tBox}
                  The \lstinline{node} definition in Jac Code~\ref{jac:addactions} is equivalent to
                  \begin{lstlisting}
node person {
has name, birthday;
can date.quantize_to_year with activity;
}
        \end{lstlisting}
                  The \lstinline{with activity} keywords indicates the action will be called by walkers.
              \end{tBox}
          \end{remark}

\end{description}


\chapterimage{chapter_head.pdf}\chapter{Navigating Graphs}% \begin{tikzpicture}[>=stealth, every node/.style={circle, draw, minimum size=0.75cm}]
%     \graph [tree layout, grow=down, fresh nodes, level distance=0.5in, sibling distance=0.5in]
%     {
%     D -> {
%     C -> { A -> { E, " " }, B,B },
%     C -> { A, B, B },
%     C -> { A, B, B }
%     }
%     };
% \end{tikzpicture}

\begin{description}
    \begin{lstlisting}[caption={Walking graphs by taking edges}]
node person: has name;

walker get_names {
    std.out(here.name)
    take -->;
}

walker build_example {
    node1 = spawn here --> node::person(name="Joe");
    node2 = spawn node1 --> node::person(name="Susan");
    spawn node2 --> node::person(name="Matt");
}

walker init {
    root {
        spawn here walker::build_example;
        take -->;
    }
    person {
        spawn here walker::get_names;
        disengage;
    }
}
    \end{lstlisting}
    \item[Output] \texttt{}
          \begin{lstlisting}[language=shell]
        \end{lstlisting}
    \item[Description] \texttt{}
\end{description}


\begin{description}
    \begin{lstlisting}[caption={Fan out style walks}]
node person: has name;

walker build_example {
    spawn here -[friend]-> node::person(name="Joe");
    spawn here -[friend]-> node::person(name="Susan");
    spawn here -[family]-> node::person(name="Matt");
}

walker init {
    root {
        spawn here walker::build_example;
        take -->;
    }
    person {
        std.out(here.name);
    }
}
    \end{lstlisting}
    \item[Output] \texttt{}
          \begin{lstlisting}[language=shell]
        \end{lstlisting}
    \item[Description] \texttt{}
\end{description}


\begin{description}
    \begin{lstlisting}[caption={Ignoring paths on a walk}]
node person: has name;
edge family;
edge friend;

walker build_example {
    spawn here -[friend]-> node::person(name="Joe");
    spawn here -[friend]-> node::person(name="Susan");
    spawn here -[family]-> node::person(name="Matt");
    spawn here -[family]-> node::person(name="Dan");
}

walker init {
    root {
        spawn here walker::build_example;
        ignore -[family]->;
        take -->;
    }
    person {
        std.out(here.name);
    }
}
    \end{lstlisting}
    \item[Output] \texttt{}
          \begin{lstlisting}[language=shell]
        \end{lstlisting}
    \item[Description] \texttt{}
\end{description}


\begin{description}
    \begin{lstlisting}[caption={Destroying (deleting) nodes}]
node person: has name;
edge family;
edge friend;

walker build_example {
    spawn here -[friend]-> node::person(name="Joe");
    spawn here -[friend]-> node::person(name="Susan");
    spawn here -[family]-> node::person(name="Matt");
    spawn here -[family]-> node::person(name="Dan");
}

walker init {
    root {
        spawn here walker::build_example;
        for i in -[friend]->: destroy i;
        take -->;
    }
    person {
        std.out(here.name);
    }
}
    \end{lstlisting}
    \item[Output] \texttt{}
          \begin{lstlisting}[language=shell]
        \end{lstlisting}
    \item[Description] \texttt{}
\end{description}


\begin{description}
    \begin{lstlisting}[caption={Generating reports throughout a walk}]
node person: has name;
edge family;
edge friend;

walker build_example {
    spawn here -[friend]-> node::person(name="Joe");
    spawn here -[friend]-> node::person(name="Susan");
    spawn here -[family]-> node::person(name="Matt");
    spawn here -[family]-> node::person(name="Dan");
}

walker init {
    root {
        spawn here walker::build_example;
        spawn -->[0] walker::build_example;
        take -->;
    }
    person {
        report here; # report print back on disengage
        take -->;
    }
}
    \end{lstlisting}
    \item[Output] \texttt{}
          \begin{lstlisting}[language=shell]
        \end{lstlisting}
    \item[Description] \texttt{}
\end{description}
\chapterimage{chapter_head.pdf}\chapter{Putting It All Together: LifeLogify}\input{tex/chap/chap-llcodedive}
\chapterimage{chapter_head.pdf}\chapter{Standard Library of Actions}
% \chapterimage{chapter_head.pdf}\chapter{Case Studies and Examples}\input{tex/chap/chap-jac-Examples}
\chapterimage{chapter_head.pdf}\chapter{Language Grammar and Specification}%%%%%%%%%%%%%%%%%%%%%%%%%%%%%%%%%%%%%%%%%%%%%%%%%%%%%%%%%%%%%%%%%%%%%%%%%%%%%%%
\begin{description}
    \item[Grammar Rules] \texttt{start, element}
          \begin{lstlisting}[style=gram]
grammar jac;

start: element*;

element: architype | walker;

        \end{lstlisting}
    \item[Example] \texttt{}
          \begin{lstlisting}
node person;
edge friend;

walker friend_network {}
        \end{lstlisting}

    \item[Description] Every program begins with a number of architypes followed by walkers.
\end{description}


%%%%%%%%%%%%%%%%%%%%%%%%%%%%%%%%%%%%%%%%%%%%%%%%%%%%%%%%%%%%%%%%%%%%%%%%%%%%%%%
\begin{description}
    \item[Grammar Rule] \texttt{architype}
          \begin{lstlisting}[style=gram]
architype:
    KW_NODE NAME (COLON INT)? attr_block
    | KW_EDGE NAME attr_block;
        \end{lstlisting}
    \item[Example] \texttt{}
          \begin{lstlisting}
        \end{lstlisting}

    \item[Description]
\end{description}


%%%%%%%%%%%%%%%%%%%%%%%%%%%%%%%%%%%%%%%%%%%%%%%%%%%%%%%%%%%%%%%%%%%%%%%%%%%%%%%
\begin{description}
    \item[Grammar Rule] \texttt{architype}
          \begin{lstlisting}[style=gram]
walker: KW_WALKER NAME LBRACE (attr_stmt)* statement* RBRACE;
        \end{lstlisting}
    \item[Example] \texttt{}
          \begin{lstlisting}
        \end{lstlisting}

    \item[Description]
\end{description}


%%%%%%%%%%%%%%%%%%%%%%%%%%%%%%%%%%%%%%%%%%%%%%%%%%%%%%%%%%%%%%%%%%%%%%%%%%%%%%%
\begin{description}
    \item[Grammar Rule] \texttt{architype}
          \begin{lstlisting}[style=gram]
attr_block:
    LBRACE (attr_stmt)* RBRACE
    | COLON (attr_stmt)* SEMI
    | SEMI;
        \end{lstlisting}
    \item[Example] \texttt{}
          \begin{lstlisting}
        \end{lstlisting}

    \item[Description]
\end{description}


%%%%%%%%%%%%%%%%%%%%%%%%%%%%%%%%%%%%%%%%%%%%%%%%%%%%%%%%%%%%%%%%%%%%%%%%%%%%%%%
\begin{description}
    \item[Grammar Rule] \texttt{architype}
          \begin{lstlisting}[style=gram]
attr_stmt: has_stmt | can_stmt;
        \end{lstlisting}
    \item[Example] \texttt{}
          \begin{lstlisting}
        \end{lstlisting}

    \item[Description]
\end{description}


%%%%%%%%%%%%%%%%%%%%%%%%%%%%%%%%%%%%%%%%%%%%%%%%%%%%%%%%%%%%%%%%%%%%%%%%%%%%%%%
\begin{description}
    \item[Grammar Rule] \texttt{architype}
          \begin{lstlisting}[style=gram]
has_stmt: KW_HAS KW_ANCHOR? NAME (COMMA NAME)* SEMI;
        \end{lstlisting}
    \item[Example] \texttt{}
          \begin{lstlisting}
        \end{lstlisting}

    \item[Description]
\end{description}


%%%%%%%%%%%%%%%%%%%%%%%%%%%%%%%%%%%%%%%%%%%%%%%%%%%%%%%%%%%%%%%%%%%%%%%%%%%%%%%
\begin{description}
    \item[Grammar Rule] \texttt{architype}
          \begin{lstlisting}[style=gram]
can_stmt:
    KW_CAN dotted_name preset_in_out? (KW_WITH KW_MOVE)? (
        COMMA dotted_name preset_in_out? (KW_WITH KW_MOVE)?
    )* SEMI;
        \end{lstlisting}
    \item[Example] \texttt{}
          \begin{lstlisting}
        \end{lstlisting}

    \item[Description]
\end{description}


%%%%%%%%%%%%%%%%%%%%%%%%%%%%%%%%%%%%%%%%%%%%%%%%%%%%%%%%%%%%%%%%%%%%%%%%%%%%%%%
\begin{description}
    \item[Grammar Rule] \texttt{architype}
          \begin{lstlisting}[style=gram]
preset_in_out: DBL_COLON NAME (COMMA NAME)* (COLON_OUT NAME)?;
        \end{lstlisting}
    \item[Example] \texttt{}
          \begin{lstlisting}
        \end{lstlisting}

    \item[Description]
\end{description}


%%%%%%%%%%%%%%%%%%%%%%%%%%%%%%%%%%%%%%%%%%%%%%%%%%%%%%%%%%%%%%%%%%%%%%%%%%%%%%%
\begin{description}
    \item[Grammar Rule] \texttt{architype}
          \begin{lstlisting}[style=gram]
code_block: LBRACE statement* RBRACE | COLON statement;
        \end{lstlisting}
    \item[Example] \texttt{}
          \begin{lstlisting}
        \end{lstlisting}

    \item[Description]
\end{description}


%%%%%%%%%%%%%%%%%%%%%%%%%%%%%%%%%%%%%%%%%%%%%%%%%%%%%%%%%%%%%%%%%%%%%%%%%%%%%%%
\begin{description}
    \item[Grammar Rule] \texttt{architype}
          \begin{lstlisting}[style=gram]
node_ctx_block: NAME (COMMA NAME)* code_block;
        \end{lstlisting}
    \item[Example] \texttt{}
          \begin{lstlisting}
        \end{lstlisting}

    \item[Description]
\end{description}


%%%%%%%%%%%%%%%%%%%%%%%%%%%%%%%%%%%%%%%%%%%%%%%%%%%%%%%%%%%%%%%%%%%%%%%%%%%%%%%
\begin{description}
    \item[Grammar Rule] \texttt{architype}
          \begin{lstlisting}[style=gram]
statement:
    code_block
    | node_ctx_block
    | expression SEMI
    | if_stmt
    | for_stmt
    | while_stmt
    | ctrl_stmt SEMI
    | action_stmt;
        \end{lstlisting}
    \item[Example] \texttt{}
          \begin{lstlisting}
        \end{lstlisting}

    \item[Description]
\end{description}


%%%%%%%%%%%%%%%%%%%%%%%%%%%%%%%%%%%%%%%%%%%%%%%%%%%%%%%%%%%%%%%%%%%%%%%%%%%%%%%
\begin{description}
    \item[Grammar Rule] \texttt{architype}
          \begin{lstlisting}[style=gram]
if_stmt: KW_IF expression code_block (elif_stmt)* (else_stmt)?;
        \end{lstlisting}
    \item[Example] \texttt{}
          \begin{lstlisting}
        \end{lstlisting}

    \item[Description]
\end{description}


%%%%%%%%%%%%%%%%%%%%%%%%%%%%%%%%%%%%%%%%%%%%%%%%%%%%%%%%%%%%%%%%%%%%%%%%%%%%%%%
\begin{description}
    \item[Grammar Rule] \texttt{architype}
          \begin{lstlisting}[style=gram]
elif_stmt: KW_ELIF expression code_block;
        \end{lstlisting}
    \item[Example] \texttt{}
          \begin{lstlisting}
        \end{lstlisting}

    \item[Description]
\end{description}


%%%%%%%%%%%%%%%%%%%%%%%%%%%%%%%%%%%%%%%%%%%%%%%%%%%%%%%%%%%%%%%%%%%%%%%%%%%%%%%
\begin{description}
    \item[Grammar Rule] \texttt{architype}
          \begin{lstlisting}[style=gram]
else_stmt: KW_ELSE code_block;
        \end{lstlisting}
    \item[Example] \texttt{}
          \begin{lstlisting}
        \end{lstlisting}

    \item[Description]
\end{description}


%%%%%%%%%%%%%%%%%%%%%%%%%%%%%%%%%%%%%%%%%%%%%%%%%%%%%%%%%%%%%%%%%%%%%%%%%%%%%%%
\begin{description}
    \item[Grammar Rule] \texttt{architype}
          \begin{lstlisting}[style=gram]
for_stmt:
    KW_FOR expression KW_TO expression KW_BY expression code_block
    | KW_FOR NAME KW_IN expression code_block;
        \end{lstlisting}
    \item[Example] \texttt{}
          \begin{lstlisting}
        \end{lstlisting}

    \item[Description]
\end{description}


%%%%%%%%%%%%%%%%%%%%%%%%%%%%%%%%%%%%%%%%%%%%%%%%%%%%%%%%%%%%%%%%%%%%%%%%%%%%%%%
\begin{description}
    \item[Grammar Rule] \texttt{architype}
          \begin{lstlisting}[style=gram]
while_stmt: KW_WHILE expression code_block;
        \end{lstlisting}
    \item[Example] \texttt{}
          \begin{lstlisting}
        \end{lstlisting}

    \item[Description]
\end{description}


%%%%%%%%%%%%%%%%%%%%%%%%%%%%%%%%%%%%%%%%%%%%%%%%%%%%%%%%%%%%%%%%%%%%%%%%%%%%%%%
\begin{description}
    \item[Grammar Rule] \texttt{architype}
          \begin{lstlisting}[style=gram]
ctrl_stmt: KW_CONTINUE | KW_BREAK | KW_DISENGAGE | KW_SKIP;
        \end{lstlisting}
    \item[Example] \texttt{}
          \begin{lstlisting}
        \end{lstlisting}

    \item[Description]
\end{description}


%%%%%%%%%%%%%%%%%%%%%%%%%%%%%%%%%%%%%%%%%%%%%%%%%%%%%%%%%%%%%%%%%%%%%%%%%%%%%%%
\begin{description}
    \item[Grammar Rule] \texttt{architype}
          \begin{lstlisting}[style=gram]
action_stmt:
    ignore_action
    | take_action
    | report_action
    | destroy_action;
        \end{lstlisting}
    \item[Example] \texttt{}
          \begin{lstlisting}
        \end{lstlisting}

    \item[Description]
\end{description}


%%%%%%%%%%%%%%%%%%%%%%%%%%%%%%%%%%%%%%%%%%%%%%%%%%%%%%%%%%%%%%%%%%%%%%%%%%%%%%%
\begin{description}
    \item[Grammar Rule] \texttt{architype}
          \begin{lstlisting}[style=gram]
ignore_action: KW_IGNORE expression SEMI;
        \end{lstlisting}
    \item[Example] \texttt{}
          \begin{lstlisting}
        \end{lstlisting}

    \item[Description]
\end{description}


%%%%%%%%%%%%%%%%%%%%%%%%%%%%%%%%%%%%%%%%%%%%%%%%%%%%%%%%%%%%%%%%%%%%%%%%%%%%%%%
\begin{description}
    \item[Grammar Rule] \texttt{architype}
          \begin{lstlisting}[style=gram]
take_action: KW_TAKE expression (SEMI | else_stmt);
        \end{lstlisting}
    \item[Example] \texttt{}
          \begin{lstlisting}
        \end{lstlisting}

    \item[Description]
\end{description}


%%%%%%%%%%%%%%%%%%%%%%%%%%%%%%%%%%%%%%%%%%%%%%%%%%%%%%%%%%%%%%%%%%%%%%%%%%%%%%%
\begin{description}
    \item[Grammar Rule] \texttt{architype}
          \begin{lstlisting}[style=gram]
report_action: KW_REPORT expression SEMI;
        \end{lstlisting}
    \item[Example] \texttt{}
          \begin{lstlisting}
        \end{lstlisting}

    \item[Description]
\end{description}


%%%%%%%%%%%%%%%%%%%%%%%%%%%%%%%%%%%%%%%%%%%%%%%%%%%%%%%%%%%%%%%%%%%%%%%%%%%%%%%
\begin{description}
    \item[Grammar Rule] \texttt{architype}
          \begin{lstlisting}[style=gram]
destroy_action: KW_DESTROY expression SEMI;
        \end{lstlisting}
    \item[Example] \texttt{}
          \begin{lstlisting}
        \end{lstlisting}

    \item[Description]
\end{description}


%%%%%%%%%%%%%%%%%%%%%%%%%%%%%%%%%%%%%%%%%%%%%%%%%%%%%%%%%%%%%%%%%%%%%%%%%%%%%%%
\begin{description}
    \item[Grammar Rule] \texttt{architype}
          \begin{lstlisting}[style=gram]
expression: assignment | connect;
        \end{lstlisting}
    \item[Example] \texttt{}
          \begin{lstlisting}
        \end{lstlisting}

    \item[Description]
\end{description}


%%%%%%%%%%%%%%%%%%%%%%%%%%%%%%%%%%%%%%%%%%%%%%%%%%%%%%%%%%%%%%%%%%%%%%%%%%%%%%%
\begin{description}
    \item[Grammar Rule] \texttt{architype}
          \begin{lstlisting}[style=gram]
assignment:
    dotted_name EQ expression
    | inc_assign
    | copy_assign;
        \end{lstlisting}
    \item[Example] \texttt{}
          \begin{lstlisting}
        \end{lstlisting}

    \item[Description]
\end{description}

%%%%%%%%%%%%%%%%%%%%%%%%%%%%%%%%%%%%%%%%%%%%%%%%%%%%%%%%%%%%%%%%%%%%%%%%%%%%%%%
\begin{description}
    \item[Grammar Rule] \texttt{architype}
          \begin{lstlisting}[style=gram]
inc_assign: dotted_name (PEQ | MEQ | TEQ | DEQ) expression;
\end{lstlisting}
    \item[Example] \texttt{}
          \begin{lstlisting}
\end{lstlisting}

    \item[Description]
\end{description}

%%%%%%%%%%%%%%%%%%%%%%%%%%%%%%%%%%%%%%%%%%%%%%%%%%%%%%%%%%%%%%%%%%%%%%%%%%%%%%%
\begin{description}
    \item[Grammar Rule] \texttt{architype}
          \begin{lstlisting}[style=gram]
copy_assign: dotted_name CPY_EQ expression;
\end{lstlisting}
    \item[Example] \texttt{}
          \begin{lstlisting}
    \end{lstlisting}

    \item[Description]
\end{description}


%%%%%%%%%%%%%%%%%%%%%%%%%%%%%%%%%%%%%%%%%%%%%%%%%%%%%%%%%%%%%%%%%%%%%%%%%%%%%%%
\begin{description}
    \item[Grammar Rule] \texttt{architype}
          \begin{lstlisting}[style=gram]
connect: logical ( (NOT)? edge_ref expression)?;
\end{lstlisting}
    \item[Example] \texttt{}
          \begin{lstlisting}
    \end{lstlisting}

    \item[Description]
\end{description}


%%%%%%%%%%%%%%%%%%%%%%%%%%%%%%%%%%%%%%%%%%%%%%%%%%%%%%%%%%%%%%%%%%%%%%%%%%%%%%%
\begin{description}
    \item[Grammar Rule] \texttt{architype}
          \begin{lstlisting}[style=gram]
logical: compare ((KW_AND | KW_OR) compare)*;
\end{lstlisting}
    \item[Example] \texttt{}
          \begin{lstlisting}
    \end{lstlisting}

    \item[Description]
\end{description}


%%%%%%%%%%%%%%%%%%%%%%%%%%%%%%%%%%%%%%%%%%%%%%%%%%%%%%%%%%%%%%%%%%%%%%%%%%%%%%%
\begin{description}
    \item[Grammar Rule] \texttt{architype}
          \begin{lstlisting}[style=gram]
compare:
	NOT compare
	| arithmetic (
		(EE | LT | GT | LTE | GTE | NE | KW_IN | nin) arithmetic
	)*;
\end{lstlisting}
    \item[Example] \texttt{}
          \begin{lstlisting}
    \end{lstlisting}

    \item[Description]
\end{description}


%%%%%%%%%%%%%%%%%%%%%%%%%%%%%%%%%%%%%%%%%%%%%%%%%%%%%%%%%%%%%%%%%%%%%%%%%%%%%%%
\begin{description}
    \item[Grammar Rule] \texttt{architype}
          \begin{lstlisting}[style=gram]
nin: NOT KW_IN;
\end{lstlisting}
    \item[Example] \texttt{}
          \begin{lstlisting}
    \end{lstlisting}

    \item[Description]
\end{description}


%%%%%%%%%%%%%%%%%%%%%%%%%%%%%%%%%%%%%%%%%%%%%%%%%%%%%%%%%%%%%%%%%%%%%%%%%%%%%%%
\begin{description}
    \item[Grammar Rule] \texttt{architype}
          \begin{lstlisting}[style=gram]
arithmetic: term ((PLUS | MINUS) term)*;
\end{lstlisting}
    \item[Example] \texttt{}
          \begin{lstlisting}
    \end{lstlisting}

    \item[Description]
\end{description}


%%%%%%%%%%%%%%%%%%%%%%%%%%%%%%%%%%%%%%%%%%%%%%%%%%%%%%%%%%%%%%%%%%%%%%%%%%%%%%%
\begin{description}
    \item[Grammar Rule] \texttt{architype}
          \begin{lstlisting}[style=gram]
term: factor ((MUL | DIV | MOD) factor)*;
\end{lstlisting}
    \item[Example] \texttt{}
          \begin{lstlisting}
    \end{lstlisting}

    \item[Description]
\end{description}


%%%%%%%%%%%%%%%%%%%%%%%%%%%%%%%%%%%%%%%%%%%%%%%%%%%%%%%%%%%%%%%%%%%%%%%%%%%%%%%
\begin{description}
    \item[Grammar Rule] \texttt{architype}
          \begin{lstlisting}[style=gram]
factor: (PLUS | MINUS) factor | power;
\end{lstlisting}
    \item[Example] \texttt{}
          \begin{lstlisting}
    \end{lstlisting}

    \item[Description]
\end{description}


%%%%%%%%%%%%%%%%%%%%%%%%%%%%%%%%%%%%%%%%%%%%%%%%%%%%%%%%%%%%%%%%%%%%%%%%%%%%%%%
\begin{description}
    \item[Grammar Rule] \texttt{architype}
          \begin{lstlisting}[style=gram]
power: func_call (POW factor)*;
\end{lstlisting}
    \item[Example] \texttt{}
          \begin{lstlisting}
    \end{lstlisting}

    \item[Description]
\end{description}


%%%%%%%%%%%%%%%%%%%%%%%%%%%%%%%%%%%%%%%%%%%%%%%%%%%%%%%%%%%%%%%%%%%%%%%%%%%%%%%
\begin{description}
    \item[Grammar Rule] \texttt{architype}
          \begin{lstlisting}[style=gram]
func_call:
	atom (LPAREN (expression (COMMA expression)*)? RPAREN)?;
\end{lstlisting}
    \item[Example] \texttt{}
          \begin{lstlisting}
    \end{lstlisting}

    \item[Description]
\end{description}


%%%%%%%%%%%%%%%%%%%%%%%%%%%%%%%%%%%%%%%%%%%%%%%%%%%%%%%%%%%%%%%%%%%%%%%%%%%%%%%
\begin{description}
    \item[Grammar Rule] \texttt{architype}
          \begin{lstlisting}[style=gram]
atom:
	INT
	| FLOAT
	| STRING
	| BOOL
	| array_ref
	| attr_ref
	| node_ref
	| edge_ref (node_ref)? /* Returns nodes even if edge */
	| list_val
	| dotted_name
	| LPAREN expression RPAREN
	| spawn;
\end{lstlisting}
    \item[Example] \texttt{}
          \begin{lstlisting}
    \end{lstlisting}

    \item[Description]
\end{description}


%%%%%%%%%%%%%%%%%%%%%%%%%%%%%%%%%%%%%%%%%%%%%%%%%%%%%%%%%%%%%%%%%%%%%%%%%%%%%%%
\begin{description}
    \item[Grammar Rule] \texttt{architype}
          \begin{lstlisting}[style=gram]
array_ref: dotted_name (LSQUARE expression RSQUARE)+;
\end{lstlisting}
    \item[Example] \texttt{}
          \begin{lstlisting}
    \end{lstlisting}

    \item[Description]
\end{description}


%%%%%%%%%%%%%%%%%%%%%%%%%%%%%%%%%%%%%%%%%%%%%%%%%%%%%%%%%%%%%%%%%%%%%%%%%%%%%%%
\begin{description}
    \item[Grammar Rule] \texttt{architype}
          \begin{lstlisting}[style=gram]
attr_ref: dotted_name DBL_COLON dotted_name;
\end{lstlisting}
    \item[Example] \texttt{}
          \begin{lstlisting}
    \end{lstlisting}

    \item[Description]
\end{description}


%%%%%%%%%%%%%%%%%%%%%%%%%%%%%%%%%%%%%%%%%%%%%%%%%%%%%%%%%%%%%%%%%%%%%%%%%%%%%%%
\begin{description}
    \item[Grammar Rule] \texttt{architype}
          \begin{lstlisting}[style=gram]
node_ref: KW_NODE (DBL_COLON NAME)?;
\end{lstlisting}
    \item[Example] \texttt{}
          \begin{lstlisting}
    \end{lstlisting}

    \item[Description]
\end{description}


%%%%%%%%%%%%%%%%%%%%%%%%%%%%%%%%%%%%%%%%%%%%%%%%%%%%%%%%%%%%%%%%%%%%%%%%%%%%%%%
\begin{description}
    \item[Grammar Rule] \texttt{architype}
          \begin{lstlisting}[style=gram]
edge_ref: edge_to | edge_from | edge_any;
\end{lstlisting}
    \item[Example] \texttt{}
          \begin{lstlisting}
    \end{lstlisting}

    \item[Description]
\end{description}


%%%%%%%%%%%%%%%%%%%%%%%%%%%%%%%%%%%%%%%%%%%%%%%%%%%%%%%%%%%%%%%%%%%%%%%%%%%%%%%
\begin{description}
    \item[Grammar Rule] \texttt{architype}
          \begin{lstlisting}[style=gram]
edge_to: '-' ('[' NAME ']')? '->';
\end{lstlisting}
    \item[Example] \texttt{}
          \begin{lstlisting}
    \end{lstlisting}

    \item[Description]
\end{description}


%%%%%%%%%%%%%%%%%%%%%%%%%%%%%%%%%%%%%%%%%%%%%%%%%%%%%%%%%%%%%%%%%%%%%%%%%%%%%%%
\begin{description}
    \item[Grammar Rule] \texttt{architype}
          \begin{lstlisting}[style=gram]
edge_from: '<-' ('[' NAME ']')? '-';
\end{lstlisting}
    \item[Example] \texttt{}
          \begin{lstlisting}
    \end{lstlisting}

    \item[Description]
\end{description}


%%%%%%%%%%%%%%%%%%%%%%%%%%%%%%%%%%%%%%%%%%%%%%%%%%%%%%%%%%%%%%%%%%%%%%%%%%%%%%%
\begin{description}
    \item[Grammar Rule] \texttt{architype}
          \begin{lstlisting}[style=gram]
edge_any: '<-' ('[' NAME ']')? '->';
\end{lstlisting}
    \item[Example] \texttt{}
          \begin{lstlisting}
    \end{lstlisting}

    \item[Description]
\end{description}


%%%%%%%%%%%%%%%%%%%%%%%%%%%%%%%%%%%%%%%%%%%%%%%%%%%%%%%%%%%%%%%%%%%%%%%%%%%%%%%
\begin{description}
    \item[Grammar Rule] \texttt{architype}
          \begin{lstlisting}[style=gram]
list_val: LSQUARE (expression (COMMA expression)*)? RSQUARE;
\end{lstlisting}
    \item[Example] \texttt{}
          \begin{lstlisting}
    \end{lstlisting}

    \item[Description]
\end{description}


%%%%%%%%%%%%%%%%%%%%%%%%%%%%%%%%%%%%%%%%%%%%%%%%%%%%%%%%%%%%%%%%%%%%%%%%%%%%%%%
\begin{description}
    \item[Grammar Rule] \texttt{architype}
          \begin{lstlisting}[style=gram]
spawn: KW_SPAWN expression spawn_object;
\end{lstlisting}
    \item[Example] \texttt{}
          \begin{lstlisting}
    \end{lstlisting}

    \item[Description]
\end{description}


%%%%%%%%%%%%%%%%%%%%%%%%%%%%%%%%%%%%%%%%%%%%%%%%%%%%%%%%%%%%%%%%%%%%%%%%%%%%%%%
\begin{description}
    \item[Grammar Rule] \texttt{architype}
          \begin{lstlisting}[style=gram]
spawn_object: node_spawn | walker_spawn;
\end{lstlisting}
    \item[Example] \texttt{}
          \begin{lstlisting}
    \end{lstlisting}

    \item[Description]
\end{description}


%%%%%%%%%%%%%%%%%%%%%%%%%%%%%%%%%%%%%%%%%%%%%%%%%%%%%%%%%%%%%%%%%%%%%%%%%%%%%%%
\begin{description}
    \item[Grammar Rule] \texttt{architype}
          \begin{lstlisting}[style=gram]
node_spawn: edge_ref node_ref spawn_ctx?;
\end{lstlisting}
    \item[Example] \texttt{}
          \begin{lstlisting}
    \end{lstlisting}

    \item[Description]
\end{description}


%%%%%%%%%%%%%%%%%%%%%%%%%%%%%%%%%%%%%%%%%%%%%%%%%%%%%%%%%%%%%%%%%%%%%%%%%%%%%%%
\begin{description}
    \item[Grammar Rule] \texttt{architype}
          \begin{lstlisting}[style=gram]
walker_spawn: KW_WALKER DBL_COLON NAME spawn_ctx?;
\end{lstlisting}
    \item[Example] \texttt{}
          \begin{lstlisting}
    \end{lstlisting}

    \item[Description]
\end{description}


%%%%%%%%%%%%%%%%%%%%%%%%%%%%%%%%%%%%%%%%%%%%%%%%%%%%%%%%%%%%%%%%%%%%%%%%%%%%%%%
\begin{description}
    \item[Grammar Rule] \texttt{architype}
          \begin{lstlisting}[style=gram]
spawn_ctx: LPAREN (assignment (COMMA assignment)*)? RPAREN;
\end{lstlisting}
    \item[Example] \texttt{}
          \begin{lstlisting}
    \end{lstlisting}

    \item[Description]
\end{description}


\part{API Substrate for Computation}\chapterimage{chapter_head.pdf}\chapter{Expressing Jaseci Computation through APIs}The primary method for executing computations using \index{Jac} and Jaseci is to submit Jac programs to a cloud service that implements the Jaseci machine.

[Benefits of doing it in the cloud, abstraction, scale, elasticity, machine configuration ease, etc]

To realize this cloud based operational and execution model a standard API interface is needed. We define this API using well established REST API principles~\cite{RESTAPIsOReilly,article_key}.


\chapterimage{chapter_head.pdf}\chapter{General Operations}\input{tex/chap/chap-api-generalops}
\chapterimage{chapter_head.pdf}\chapter{Jac Execution API}\input{tex/chap/chap-jac-api}
\chapterimage{chapter_head.pdf}\chapter{Jaseci Object API}\input{tex/chap/chap-obj-api}
\part{Case Studies}\chapterimage{chapter_head.pdf}\chapter{LifeLogify}\input{tex/chap/chap-case-ll}
\chapterimage{chapter_head.pdf}\chapter{FPWSM}\input{tex/chap/chap-case-fpwsm}
\chapterimage{chapter_head.pdf}\chapter{Conversational AI}\input{tex/chap/chap-case-convai}


This statement requires citation \cite{article_key}; this one is more specific \cite[162]{book_key}.

\appendix\part{Appendix}
\chapterimage{chapter_head.pdf}\chapter{Jaseci Shell}\input{tex/chap/chap-shell}
\chapterimage{chapter_head.pdf}\chapter{Jac Language Grammar}\begin{lstlisting}[style=gram, numbers=none,
    captionpos=t, caption=Jac Language Grammar]
grammar jac;

/* Sentinels handle these top rules */
start: element*;

element: architype | walker;

architype:
    KW_NODE NAME (COLON INT)? attr_block
    | KW_EDGE NAME attr_block;

walker: KW_WALKER NAME LBRACE (attr_stmt)* statement* RBRACE;

attr_block:
    LBRACE (attr_stmt)* RBRACE
    | COLON (attr_stmt)* SEMI
    | SEMI;

attr_stmt: has_stmt | can_stmt;

has_stmt: KW_HAS KW_ANCHOR? NAME (COMMA NAME)* SEMI;

can_stmt:
    KW_CAN dotted_name preset_in_out? (KW_WITH KW_MOVE)? (
        COMMA dotted_name preset_in_out? (KW_WITH KW_MOVE)?
    )* SEMI;

preset_in_out: DBL_COLON NAME (COMMA NAME)* (COLON_OUT NAME)?;

dotted_name: NAME (DOT NAME)*;

code_block: LBRACE statement* RBRACE | COLON statement;

node_ctx_block: NAME (COMMA NAME)* code_block;

statement:
    code_block
    | node_ctx_block
    | expression SEMI
    | if_stmt
    | for_stmt
    | while_stmt
    | ctrl_stmt SEMI
    | action_stmt;

if_stmt: KW_IF expression code_block (elif_stmt)* (else_stmt)?;

elif_stmt: KW_ELIF expression code_block;

else_stmt: KW_ELSE code_block;

for_stmt:
    KW_FOR expression KW_TO expression KW_BY expression code_block;

while_stmt: KW_WHILE expression code_block;

ctrl_stmt: KW_CONTINUE | KW_BREAK | KW_DISENGAGE;

action_stmt: ignore_action | take_action | report_action;

ignore_action: KW_IGNORE expression SEMI;

take_action: KW_TAKE expression (SEMI | else_stmt);

report_action: KW_REPORT expression SEMI;

expression: assignment | connect;

assignment:
    dotted_name EQ expression
    | inc_assign
    | copy_assign;

inc_assign: dotted_name (PEQ | MEQ | TEQ | DEQ) expression;

copy_assign: dotted_name CPY_EQ expression;

connect: logical (edge_ref expression)?;

logical: compare ((KW_AND | KW_OR) compare)*;

compare:
    NOT compare
    | arithmetic ((EE | LT | GT | LTE | GTE | NE) arithmetic)*;

arithmetic: term ((PLUS | MINUS) term)*;

term: factor ((MUL | DIV) factor)*;

factor: (PLUS | MINUS) factor | power;

power: func_call (POW factor)*;

func_call:
    atom (LPAREN (expression (COMMA expression)*)? RPAREN)?;

atom:
    INT
    | FLOAT
    | STRING
    | array_ref
    | attr_ref
    | node_ref
    | edge_ref (node_ref)?
    | list_val
    | dotted_name
    | LPAREN expression RPAREN
    | spawn;

array_ref: dotted_name (LSQUARE expression RSQUARE)+;

attr_ref: dotted_name DBL_COLON dotted_name;

node_ref: KW_NODE (DBL_COLON NAME)?;

edge_ref: edge_to | edge_from | edge_any;

edge_to: '-' ('[' NAME ']')? '->';

edge_from: '<-' ('[' NAME ']')? '-';

edge_any: '<-' ('[' NAME ']')? '->';

list_val: LSQUARE (expression (COMMA expression)*)? RSQUARE;

spawn: KW_SPAWN expression spawn_object;

spawn_object: node_spawn | walker_spawn;

node_spawn: edge_ref node_ref spawn_ctx?;

walker_spawn: KW_WALKER DBL_COLON NAME spawn_ctx?;

spawn_ctx: LPAREN (assignment (COMMA assignment)*)? RPAREN;
\end{lstlisting}


\begin{lstlisting}[style=gram, numbers=none,
    captionpos=t, caption=Jac Language Lexer Rules]
/* Lexer rules */
KW_NODE: 'node';
KW_IGNORE: 'ignore';
KW_TAKE: 'take';
KW_MOVE: 'entry' | 'activity' | 'exit';
KW_SPAWN: 'spawn';
KW_WITH: 'with';
COLON: ':';
DBL_COLON: '::';
COLON_OUT: '::>';
LBRACE: '{';
RBRACE: '}';
KW_EDGE: 'edge';
KW_WALKER: 'walker';
SEMI: ';';
EQ: '=';
PEQ: '+=';
MEQ: '-=';
TEQ: '*=';
DEQ: '/=';
CPY_EQ: ':=';
KW_AND: 'and' | '&&';
KW_OR: 'or' | '||';
KW_IF: 'if';
KW_ELIF: 'elif';
KW_ELSE: 'else';
KW_FOR: 'for';
KW_TO: 'to';
KW_BY: 'by';
KW_WHILE: 'while';
KW_CONTINUE: 'continue';
KW_BREAK: 'break';
KW_DISENGAGE: 'disengage';
KW_REPORT: 'report';
DOT: '.';
NOT: '!';
EE: '==';
LT: '<';
GT: '>';
LTE: '<=';
GTE: '>=';
NE: '!=';
KW_ANCHOR: 'anchor';
KW_HAS: 'has';
COMMA: ',';
KW_CAN: 'can';
PLUS: '+';
MINUS: '-';
MUL: '*';
DIV: '/';
POW: '^';
LPAREN: '(';
RPAREN: ')';
LSQUARE: '[';
RSQUARE: ']';
FLOAT: [0-9]+ '.' [0-9]+;
STRING: '"' ~ ["\r\n]* '"' | '\'' ~ ['\r\n]* '\'';
INT: [0-9]+;
NAME: [a-zA-Z_] [a-zA-Z0-9_]*;
COMMENT: '/*' .*? '*/' -> skip;
LINE_COMMENT: '//' ~[\r\n]* -> skip;
PY_COMMENT: '#' ~[\r\n]* -> skip;
WS: [ \t\r\n] -> skip;
ErrorChar: .;
\end{lstlisting}

%----------------------------------------------------------------------------------------
%	BIBLIOGRAPHY
%----------------------------------------------------------------------------------------

\chapter*{Bibliography}
\addcontentsline{toc}{chapter}{\textcolor{ocre}{Bibliography}} % Add a Bibliography heading to the table of contents

%------------------------------------------------

\section*{Articles}
\addcontentsline{toc}{section}{Articles}
\printbibliography[heading=bibempty,type=article]

%------------------------------------------------

\section*{Books}
\addcontentsline{toc}{section}{Books}
\printbibliography[heading=bibempty,type=book]


%----------------------------------------------------------------------------------------
%	INDEX
%----------------------------------------------------------------------------------------

\cleardoublepage % Make sure the index starts on an odd (right side) page
\phantomsection
\setlength{\columnsep}{0.75cm} % Space between the 2 columns of the index
\addcontentsline{toc}{chapter}{\textcolor{ocre}{Index}} % Add an Index heading to the table of contents
\printindex % Output the index

%----------------------------------------------------------------------------------------

\end{document}
