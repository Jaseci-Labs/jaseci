\chapter{Interfacing a Jaseci Machine}
\minitoc
\jacdot{dia_api_server_client}{.6}{Jaseci Interface Architecture}
Now that we know what Jaseci is all about, next lets roll up our sleeves and jump in. One of the best ways to jump into Jaseci world is to gather some sample Jac programs and start tinkering with them.
\par
Before we jump right into it, it's important to have a bit of an understanding of the the way the interface itself is architected from in the implementation of the Jaseci stack. Jaseci has a module that serves as its  the core interface (summarized in Table~\ref{tab:jsAPI}) to the Jaseci machine. This interface is expressed as a set of method functions within a python class in Jaseci  called \texttt{master}. (By the way, don't worry, it's ok to use ``master'', its not racialist, see Rant~\ref{rant:racistmaster} for more context). The `client' expressions of that interface in the forms of a command line tool \texttt{jsctl} and a server-side REST API built using Django~\footnote{Django ~\cite{django} is a Python web framework for rapid development and clean, pragmatic design}. Figure~\ref{dot:dia_api_server_client} illustrates this architecture representing the relationship between core APIs and client side expressions.

If I may say so myself the code architecture of interface generation from function signatures is elegant, sexy, and takes advantage of the best python has to offer in terms of its support for introspection. With this approach, as the set of functions and their semantics change in the \texttt{master} API class, both the JSCTL Cli tool and the REST Server-side API changes. We dig into this and tons more in the Part~\ref{part:crafting}, so we'll leave the discussion on implementation architecture there for the moment. Lets jump right into how we get started playing with some \gls{leet} Jaseci \gls{haxor}ing. First we start with JSCTL then dive into the REST API.


\section{JSCTL: The Jaseci Command Line Interface}
JSCTL or \texttt{jsctl} is a command line tool that provides full access to Jaseci. This tool is installed alongside the installation of the Jaseci Core package and should be accessible from the command line from anywhere. Let's say you've just checked out the Jaseci repo and you're in head folder. You should be able to execute the following.
\par
\shellout{jsctl_setup.shell}
\par
Here we've installed the Jaseci python package that can be imported into any python project with a directive such as \texttt{import jaseci}, and at the same time, we've installed the \texttt{jsctl} command line tool into our OS environment. At this point we can issue a call to say \texttt{jsctl --help} for any working directory.
\begin{nerd}
    Python Code~\ref{py:setup.py} shows the implementation of \texttt{setup.py} that is responsible for deploying the jsctl tool upon \texttt{pip3} installation of Jaseci Core.
    \pycode{setup.py}{setup.py for Jaseci Core}
\end{nerd}

\subsection{The Very Basics: CLI vs Shell-mode, and Session Files }
This command line tool provides full access to the Jaseci core APIs via the command line, or a shell mode. In shell mode, all of the same Jaseci API functionally is available within a single session. To invoke shell-mode, simply execute \texttt{jsctl} without any commands and jsctl will enter shell mode as per the example below.
\par
\shellout{jsctl_shell_mode.shell}
\par
Here we launched \texttt{jsctl} directly into shell mode for a single session and we can issue various calls to the Jaseci API for that session. In this example we issue a single call to \texttt{graph create}, which creates a graph within the Jaseci session with a single root node, then exit the shell with \texttt{exit}.
\par
The exact behavior can be achieved without ever entering the shell directly from the command line as shown below.
\par
\shellout{jsctl_cli_mode.shell}
\par
All such calls to Jaseci's API (summarized in Table~\ref{tab:jsAPI}) can be issued either through shell-mode and CLI mode.
\paragraph{Session Files}
At this point, it's important to understand how sessions work. In a nutshell, a session captures the complete state of a jaseci machine. This state includes the status of memory, graphs, walkers, configurations, etc. The complete state of a Jaseci machine can be captured in a \texttt{.session} file. Every time state changes for a given session via the \texttt{jsctl} tool the assigned session file is updated. If you've been following along so far, try this.
\par
\shellout{session_default.shell}
\par
Note from the first call to \texttt{ls} we have a session file that has been created call \texttt{js.session}. This is the default session file \texttt{jsctl} creates and utilizes when called either in cli mode or shell mode. After listing session files, notices the call to \texttt{graph list} which lists the root nodes of all graphs created within a Jaseci machine's state. Note \texttt{jsctl} lists two such graph root nodes. Indeed these nodes correspond to the ones we've just created when contrasting cli mode and shell mode above. Having these two graphs demonstrates that across both instantiations of \texttt{jsctl} the same session, \texttt{js.session}, is being used. Now try the following.
\par
\shellout{new_session.shell}
\par
Here we see that we can use the \texttt{-f} or \texttt{--filename} flag to specify the session file to use. In this case we list the graphs of the session corresponding to \texttt{mynew.session} and see the JSON representation of an empty list of objects. We then list session files and see that one was created for \texttt{mynew.session}. If we were to now type \texttt{jsctl --filename js.session graph list}, we would see a list of the two graph objects that we created earlier.
\paragraph{In-memory mode}
Its important to note that there is also an in-memory mode that can be created buy using the \texttt{-m} or \texttt{--mem-only} flags. This flag is particularly useful when you'd simply like to tinker around with a machine in shell-mode or you'd like to script some behavior to be executed in Jac and have no need to maintain machine state after completion. We will be using in memory session mode quite a bit, so you'll get a sense of its usage throughout this chapter. Next we actually see a workflow for tinkering.

\subsection{A Simple Workflow for Tinkering}

As you get to know Jaseci and Jac, you'll want to try things and tinker a bit. In this section, we'll get to know how \texttt{jsctl} can be used as the main platform for this play. A typical flow will involve jumping into shell-mode, writing some code, running that code to observe output, and in visualizing the state of the graph, and rendering that graph in dot to see it's visualization.
\paragraph{Install Graphvis}
Before we jump right in, let me strongly encourage you install Graphviz. Graphviz is open source graph visualization software package that includes a handy dandy command line tool call \texttt{dot}. Dot is also a standardized and open graph description language that is a key primitive of Graphviz. The \texttt{dot} tool in Graphviz takes dot code and renders it nicely. Graphviz is super easy to install. In Ubuntu simply type \texttt{sudo apt install graphviz}, or on mac type \texttt{brew install graphviz} and you're done! You should be able to call \texttt{dot} from the command line.
\par
Ok, lets start with a scenario. Say you'd like to write your first Jac program which will include some nodes, edges, and walkers and you'd like to print to standard output and see what the graph looks like after you run an interesting walker. Let role play.
\par
Lets hop into a \texttt{jsctl} shell.
\par
\shellout{wf_init.shell}
\par
Good, we're in! And we've set the session to be an in-memory session so no session file will be created or saved. For this play session we only care about the Jac program we write, which will be saved. The state of the Jaseci machine we run our toy program on doesn't really matter to us.
\par
Now that we've got our shell running, we first want to create a blank graph. Remember, all walkers, Jaseci's primary unit of computation, must run on a node. As default, we can use the root node of a freshly created graph, hence we need to create a base graph. But oh no! We're a bit rusty and have forgotten how create our initial graph using \texttt{jsctl}. Let's navigate the help menu to jog our memories.
\par
\shellout{wf_help_nav.shell}
\par
Ohhh yeah! That's it. After simply using \texttt{help} from the shell we were able to navigate to the relevant info for \texttt{graph create}. Let's use this newly gotten wisdom.
\par
\shellout{wf_graph.shell}
\par
Great! With this command a graph is created and a single root node is born. \texttt{jsctl} shares with us the details of this root graph node. In Jaseci, graphs are referenced by their root nodes and every graph has a single root node.
\par
Notice we've also set the \texttt{-set\_active} parameter to true. This parameter informs Jaseci to use the root node of this graph (in particular the UUID of this root node) as the default parameter to all future calls to Jaseci Core APIs that have a parameter specifying a graph or node to operate on. This global designation that this graph is the `active' graph is a convenience feature so we the user doesn't have to specify this parameter for future calls. Of course this can be overridden, more on that later.
\par
Next, lets write some Jac code for our little program. \texttt{jsctl} has a built in editor that is simple yet powerful. You can use either this built in editor, or your favorite editor to create the \texttt{.jac} file for our toy program. Let's use the built in editor.
\par
\shellout{wf_edit_jac.shell}
\par
The \texttt{edit} command invokes the built in editor. Though it's a terminal editor based on \texttt{ncurses}, you can basically use it much like you'd use any wysiwyg editor with features like standard cut \texttt{ctrl-c} and paste \texttt{ctrl-v}, mouse text selection, etc. It's based on the phenomenal pure python project from Google called \texttt{ci\_edit}. For more detailed help cheat sheet see Appendix. If you must use your own favorite editor, simply be sure that you save the fam.jac file in the same working directory from which you are running the Jaseci shell. Now type out the toy program in Jac Code~\ref{jac:fam.jac}.
\par
\jaccode{fam.jac}{Jac Family Toy Program}
\par
Don't worry if that looks like the most cryptic \gls{gobbledygook} you've ever seen in your life. As you learn the Jac language, all will become clear. For now, lets tinker around. Now save and quit the editor. If you are using the built in editor thats simply a \texttt{ctrl-s, ctrl-q} combo.
\par
Ok, now we should have a \texttt{fam.jac} file saved in our working directory. We can check from the Jaseci shell!
\par
\shellout{wf_ls.shell}
\par
We can list files from the shell prompt. By default the \texttt{ls} command only lists files relevant to Jaseci (i.e., \texttt{*.jac}, \texttt{*.dot}, etc). To list all files simply add a \texttt{--all} or \texttt{-a}.
\par
Now, on to what is on of the key operations. Lets ``register'' a \gls{sentinel} based on our Jac program. A sentinel is the abstraction Jaseci uses to encapsulate compiled walkers and architype nodes and edges. You can think of registering a sentinel as compiling your jac program. The walkers of a given sentinel can then be invoked and run on arbitrary nodes of any graph. Let's register our Jac toy program.
\par
\shellout{wf_sent_reg.shell}
\par
Ok, theres a lot that just happened there. First, we see some logging output that informs us that the Jac code is being processed (which really means the Jac program is being parsed and IR being generated). If there are any syntax errors or other issues, this is where the error output will be printed along with any problematic lines of code and such. If all goes well, we see the next logging output that the code has been successfully registered. The formal output is the relevant details of the successfully created sentinel. Note, that we've also made this the ``active'' sentinel meaning it will be used as the default setting for any calls to Jaseci Core APIs that require a sentinel be specified. At this point, Jaseci has registered our code and we are ready to run walkers!
\par
But first, lets take a quick look at some of the objects loaded into our Jaseci machine. For this I'll briefly introduce the \texttt{alias} group of APIs.
\par
\shellout{wf_aliases.shell}
\par
The \texttt{alias} set of APIs are designed as an additional set of convenience tools to simplify the referencing of various objects (walkers, architypes, etc) in Jaseci. Instead of having to use the UUIDs to reference each object, an alias can be used to refer to any object. These aliases can be created or removed utilizing the \texttt{alias} APIs.
\par
Upon registering a sentinel, a set of aliases are automatically created for each object produced from processing the corresponding Jac program. The call to \texttt{alias list} lists all available aliases in the session. Here, we're using this call to see the objects that were created for our toy program and validate it corresponds to the ones we would expect from the Jac Program represented in JC~\ref{jac:fam.jac}. Everything looks good!
\par
Now, for the big moment! lets run our walker on the root node of the graph we created and see what happens!
\par
\shellout{wf_run_walker.shell}
\par
Sweet!! We see the standard output we'd expect from our toy program. Hrm, as we'd expect, when it comes to the family, the man doesn't do much it seems.
\par
But there were many semantics to what our toy program does. How do we visualize that the graph produced by or program is right. Well we're in luck! We can use Jaseci `dot' features to take a look at our graph!!
\par
\shellout{wf_dot_render.shell}
\par
\jacdot{fam}{.3}{Graph for \texttt{fam.jac}}
Here we've used the \texttt{graph get} core API to get a print out of the graph in dot format. By default \texttt{graph get} dumps out a list of all edge and node objects of the graph, however with the \texttt{-mode dot} parameter we've specified that the graph should be printed in dot. The \texttt{-o} flag specifies a file to dump the output of the command. Note that the \texttt{-o} flag for \texttt{jsctl} commands only outputs the formal returned data (json payload, or string) from a Jaseci Core API. Logging output, standard output, etc will not be saved to the file though anything reported by a walker using \texttt{report} will be saved. This output file directive is \texttt{jsctl} specific and work with any command given to \texttt{jsctl}.
\par
To see a pretty visual of the graph itself, we can use the \texttt{dot} command from Graphviz. Simply type \texttt{dot -Tpdf fam.dot -o fam.pdf} and Voila! We can see the beautiful graph our toy Jac program has produced on its way to the standard output.
\par
Awesomeness! We are Jac \Gls{haxor}s now!

\section{Jaseci REST API}
\subsection{API Parameter Cheatsheet}
\printtabJSAPI
\section{Full Spec of Jaseci Core APIs}
\subsection{APIs for actions}

\par
This set action APIs enable the manual management of Jaseci actions and action
libraries/sets. Action libraries can be loaded locally into the running instance of
the python program, or as a remote container linked action library. In this mode,
action libraries operate as micro-services. Jaseci will be able to dynamically
and automatically make this decision for the user based on online monitoring and
performance profiling.

\subsubsection{\lstinline[basicstyle=\Large\ttfamily]$actions load local$}

\apispec{cli: actions load local | api: actions\_load\_local | auth: admin}{file: str (*req)}
{This API will dynamically load a module based on a python file. The module
is loaded directly into the running Jaseci python instance. This API also
makes an attempt to auto detect and hot load any python package dependencies
the file may reference via python's relative imports. This file is assumed to
have the necessary annotations and decorations required by Jaseci to recognize
its actions.\vspace{4mm}\par
\argspec{Parameters}{
\texttt{file} -- The python file with full to load actions from.
(i.e., ~/local/myact.py)\vspace{1.5mm}\par
}}
\subsubsection{\lstinline[basicstyle=\Large\ttfamily]$actions load remote$}

\apispec{cli: actions load remote | api: actions\_load\_remote | auth: admin}{url: str (*req)}
{This API will dynamically load a set of actions that are present on a remote
server/micro-service. This server must be configured to interact with Jaseci
properly. This is easily achieved using the same decorators used for local
action libraries. Remote actions allow for higher flexibility in the languages
supported for action libraries. If an  library writer would like to use another
language, the main hook REST api simply needs to be implemented. Please
refer to documentation on creating action libraries for more details.\vspace{4mm}\par
\argspec{Parameters}{
\texttt{url} -- The url of the API server supporting Jaseci actions.\vspace{1.5mm}\par
}}
\subsubsection{\lstinline[basicstyle=\Large\ttfamily]$actions load module$}

\apispec{cli: actions load module | api: actions\_load\_module | auth: admin}{mod: str (*req)}
{This API will dynamically load a module using python's module import format.
This is particularly useful for pip installed action libraries as the developer
can directly reference the module using the same format as a regular python
import. As with load local, the module will be loaded directly into the running
Jaseci python instance.\vspace{4mm}\par
\argspec{Parameters}{
\texttt{mod} -- The import style module to load actions from.
(i.e., jaseci\_kit.bi\_enc)\vspace{1.5mm}\par
}}
\subsubsection{\lstinline[basicstyle=\Large\ttfamily]$actions list$}

\apispec{cli: actions list | api: actions\_list | auth: admin}{name: str ()}
{This API is used to list the loaded actions active in Jaseci. These actions
include all types of loaded actions whether it be local modules or remote
containers. A particular set of actions can be viewed using the name parameter.\vspace{4mm}\par
\argspec{Parameters}{
\texttt{name} -- The name for a library for which to filter the view of shown
actions. If left blank all actions from all loaded sets will be shown.\vspace{1.5mm}\par
}}
\subsection{APIs for alias}

\par
The alias set of APIs provide a set of `alias' management functions for
creating and managing aliases for long strings such as UUIDs. If an alias'
name is used as a parameter value in any API call, that parameter will see
the alias' value instead. Given that references to all sentinels, walkers,
nodes, etc. utilize UUIDs, it becomes quite useful to create pneumonic
names for them. Also, when registering   sentinels, walkers, architype
handy aliases are automatically generated. These generated aliases can
then be managed using the alias APIs. Keep in mind that whenever an alias
is created, all parameter values submitted to any API with the alias name
will be replaced internally by its value. If you get in a bind, simply use
the clear or delete alias APIs.

\subsubsection{\lstinline[basicstyle=\Large\ttfamily]$alias register$}

\apispec{cli: alias register | api: alias\_register | auth: user}{name: str (*req), value: str (*req)}
{Either create new alias string to string mappings or replace
an existing mappings of a given alias name. Once registered the
alias mapping is instantly active.\vspace{4mm}\par
\argspec{Parameters}{
\texttt{name} -- The name for the alias created by caller.\vspace{1.5mm}\par

\texttt{value} -- The value for that name to map to (i.e., UUID)\vspace{1.5mm}\par
}\vspace{4mm}\par
\argspec{Returns}{Fields include
'response': Message of mapping that was created}}
\subsubsection{\lstinline[basicstyle=\Large\ttfamily]$alias list$}

\apispec{cli: alias list | api: alias\_list | auth: user}{n/a}
{Returns dictionary object of name to value mappings currently active.
This API is quite useful to track not only the aliases the caller
creates, but also the aliases automatically created as various Jaseci
objects (walkers, architypes, sentinels, etc.) are created, changed,
or destroyed.\vspace{4mm}\par
\argspec{Returns}{Dictionary of active mappings
'name': 'value'
...}}
\subsubsection{\lstinline[basicstyle=\Large\ttfamily]$alias delete$}

\apispec{cli: alias delete | api: alias\_delete | auth: user}{name: str (*req)}
{Removes a specific alias by its name. Only the alias is removed no
actual objects are affected. Future uses of this name will not be
mapped.\vspace{4mm}\par
\argspec{Parameters}{
\texttt{name} -- The name for the alias to be removed from caller.\vspace{1.5mm}\par
}\vspace{4mm}\par
\argspec{Returns}{Fields include
'response': Message of success/failure to remove alias
'success': True/False based on delete actually happening}}
\subsubsection{\lstinline[basicstyle=\Large\ttfamily]$alias clear$}

\apispec{cli: alias clear | api: alias\_clear | auth: user}{n/a}
{Removes a all aliases. No actual objects are affected. Aliases will
continue to be automatically generated when creating other Jaseci
objects.\vspace{4mm}\par
\argspec{Returns}{Fields include
'response': Message of number of alias removed
'removed': Number of aliases removed}}
\subsection{APIs for architype}

\par
The architype set of APIs allow for the addition and removing of
architypes. Given a Jac implementation of an architype these APIs are
designed for creating, compiling, and managing architypes that can be
used by Jaseci. There are two ways to add an architype to Jaseci, either
through the management of sentinels using the sentinel API, or by
registering independent architypes with these architype APIs. These
APIs are also used for inspecting and managing existing arichtypes that
a Jaseci instance is aware of.

\subsubsection{\lstinline[basicstyle=\Large\ttfamily]$architype register$}

\apispec{cli: architype register | api: architype\_register | auth: user}{code: str (*req), encoded: bool (False), snt: sentinel (None)}
{This register API allows for the creation or replacement/update of
an architype that can then be used by walkers in their interactions
of graphs. The code argument takes Jac source code for the single
architype. To load multiple architypes and walkers at the same time,
use sentinel register API.\vspace{4mm}\par
\argspec{Parameters}{
\texttt{code} -- The text (or filename) for an architypes Jac code\vspace{1.5mm}\par

\texttt{encoded} -- True/False flag as to whether code is encode
in base64\vspace{1.5mm}\par

\texttt{snt} -- The UUID of the sentinel to be the owner of this
architype\vspace{1.5mm}\par
}\vspace{4mm}\par
\argspec{Returns}{Fields include
'architype': Architype object if created otherwise null
'success': True/False whether register was successful
'errors': List of errors if register failed
'response': Message on outcome of register call}}
\subsubsection{\lstinline[basicstyle=\Large\ttfamily]$architype get$}

\apispec{cli: architype get | api: architype\_get | auth: user}{arch: architype (*req), mode: str (default), detailed: bool (False)}
{No documentation yet.\vspace{4mm}\par
\argspec{Parameters}{
\texttt{arch} -- The architype being accessed\vspace{1.5mm}\par

\texttt{mode} -- Valid modes: {default, code, ir, }\vspace{1.5mm}\par

\texttt{detailed} -- Flag to give summary or complete set of fields\vspace{1.5mm}\par
}\vspace{4mm}\par
\argspec{Returns}{Fields include (depends on mode)
'code': Formal source code for architype
'ir': Intermediate representation of architype
'architype': Architype object print}}
\subsubsection{\lstinline[basicstyle=\Large\ttfamily]$architype set$}

\apispec{cli: architype set | api: architype\_set | auth: user}{arch: architype (*req), code: str (*req), mode: str (default)}
{No documentation yet.\vspace{4mm}\par
\argspec{Parameters}{
\texttt{arch} -- The architype being set\vspace{1.5mm}\par

\texttt{code} -- The text (or filename) for an architypes Jac code/ir\vspace{1.5mm}\par

\texttt{mode} -- Valid modes: {default, code, ir, }\vspace{1.5mm}\par
}\vspace{4mm}\par
\argspec{Returns}{Fields include (depends on mode)
'success': True/False whether set was successful
'errors': List of errors if set failed
'response': Message on outcome of set call}}
\subsubsection{\lstinline[basicstyle=\Large\ttfamily]$architype list$}

\apispec{cli: architype list | api: architype\_list | auth: user}{snt: sentinel (None), detailed: bool (False)}
{No documentation yet.\vspace{4mm}\par
\argspec{Parameters}{
\texttt{snt} -- The sentinel for which to list its architypes\vspace{1.5mm}\par

\texttt{detailed} -- Flag to give summary or complete set of fields\vspace{1.5mm}\par
}\vspace{4mm}\par
\argspec{Returns}{List of architype objects}}
\subsubsection{\lstinline[basicstyle=\Large\ttfamily]$architype delete$}

\apispec{cli: architype delete | api: architype\_delete | auth: user}{arch: architype (*req), snt: sentinel (None)}
{No documentation yet.\vspace{4mm}\par
\argspec{Parameters}{
\texttt{arch} -- The architype being set\vspace{1.5mm}\par

\texttt{snt} -- The sentinel for which to list its architypes\vspace{1.5mm}\par
}\vspace{4mm}\par
\argspec{Returns}{Fields include (depends on mode)
'success': True/False whether command was successful
'response': Message on outcome of command}}
\subsection{APIs for config}

Abstracted since there are no valid configs in core atm, see jaseci\_serv
to see how used.

\subsubsection{\lstinline[basicstyle=\Large\ttfamily]$config get$}

\apispec{cli: config get | api: config\_get | auth: admin}{name: str (*req), do_check: bool (True)}
{No documentation yet.}
\subsubsection{\lstinline[basicstyle=\Large\ttfamily]$config set$}

\apispec{cli: config set | api: config\_set | auth: admin}{name: str (*req), value: str (*req), do_check: bool (True)}
{No documentation yet.}
\subsubsection{\lstinline[basicstyle=\Large\ttfamily]$config list$}

\apispec{cli: config list | api: config\_list | auth: admin}{n/a}
{No documentation yet.}
\subsubsection{\lstinline[basicstyle=\Large\ttfamily]$config index$}

\apispec{cli: config index | api: config\_index | auth: admin}{n/a}
{No documentation yet.}
\subsubsection{\lstinline[basicstyle=\Large\ttfamily]$config exists$}

\apispec{cli: config exists | api: config\_exists | auth: admin}{name: str (*req)}
{No documentation yet.}
\subsubsection{\lstinline[basicstyle=\Large\ttfamily]$config delete$}

\apispec{cli: config delete | api: config\_delete | auth: admin}{name: str (*req), do_check: bool (True)}
{No documentation yet.}
\subsection{APIs for global}

No documentation yet.

\subsubsection{\lstinline[basicstyle=\Large\ttfamily]$global set$}

\apispec{cli: global set | api: global\_set | auth: admin}{name: str (*req), value: str (*req)}
{No documentation yet.}
\subsubsection{\lstinline[basicstyle=\Large\ttfamily]$global delete$}

\apispec{cli: global delete | api: global\_delete | auth: admin}{name: str (*req)}
{No documentation yet.}
\subsubsection{\lstinline[basicstyle=\Large\ttfamily]$global sentinel set$}

\apispec{cli: global sentinel set | api: global\_sentinel\_set | auth: admin}{snt: sentinel (None)}
{No documentation yet.}
\subsubsection{\lstinline[basicstyle=\Large\ttfamily]$global sentinel unset$}

\apispec{cli: global sentinel unset | api: global\_sentinel\_unset | auth: admin}{n/a}
{No documentation yet.}
\subsection{APIs for graph}

No documentation yet.

\subsubsection{\lstinline[basicstyle=\Large\ttfamily]$graph create$}

\apispec{cli: graph create | api: graph\_create | auth: user}{set_active: bool (True)}
{No documentation yet.}
\subsubsection{\lstinline[basicstyle=\Large\ttfamily]$graph get$}

\apispec{cli: graph get | api: graph\_get | auth: user}{gph: graph (None), mode: str (default), detailed: bool (False)}
{Valid modes: {default, dot, }}
\subsubsection{\lstinline[basicstyle=\Large\ttfamily]$graph list$}

\apispec{cli: graph list | api: graph\_list | auth: user}{detailed: bool (False)}
{No documentation yet.}
\subsubsection{\lstinline[basicstyle=\Large\ttfamily]$graph active set$}

\apispec{cli: graph active set | api: graph\_active\_set | auth: user}{gph: graph (*req)}
{No documentation yet.}
\subsubsection{\lstinline[basicstyle=\Large\ttfamily]$graph active unset$}

\apispec{cli: graph active unset | api: graph\_active\_unset | auth: user}{n/a}
{No documentation yet.}
\subsubsection{\lstinline[basicstyle=\Large\ttfamily]$graph active get$}

\apispec{cli: graph active get | api: graph\_active\_get | auth: user}{detailed: bool (False)}
{No documentation yet.}
\subsubsection{\lstinline[basicstyle=\Large\ttfamily]$graph delete$}

\apispec{cli: graph delete | api: graph\_delete | auth: user}{gph: graph (*req)}
{No documentation yet.}
\subsubsection{\lstinline[basicstyle=\Large\ttfamily]$graph node get$}

\apispec{cli: graph node get | api: graph\_node\_get | auth: user}{nd: node (*req), keys: list ([])}
{No documentation yet.}
\subsubsection{\lstinline[basicstyle=\Large\ttfamily]$graph node set$}

\apispec{cli: graph node set | api: graph\_node\_set | auth: user}{nd: node (*req), ctx: dict (*req), snt: sentinel (None)}
{No documentation yet.}
\subsubsection{\lstinline[basicstyle=\Large\ttfamily]$graph walk$}

\apispec{cli: graph walk (cli only)}{nd: node (None)}
{No documentation yet.}
\subsection{APIs for jac}

No documentation yet.

\subsubsection{\lstinline[basicstyle=\Large\ttfamily]$jac build$}

\apispec{cli: jac build (cli only)}{file: str (*req), out: str ()}
{No documentation yet.}
\subsubsection{\lstinline[basicstyle=\Large\ttfamily]$jac test$}

\apispec{cli: jac test (cli only)}{file: str (*req), detailed: bool (False)}
{and .jir executables}
\subsubsection{\lstinline[basicstyle=\Large\ttfamily]$jac run$}

\apispec{cli: jac run (cli only)}{file: str (*req), walk: str (init), ctx: dict (\{\}), profiling: bool (False)}
{and .jir executables}
\subsubsection{\lstinline[basicstyle=\Large\ttfamily]$jac dot$}

\apispec{cli: jac dot (cli only)}{file: str (*req), walk: str (init), ctx: dict (\{\}), detailed: bool (False)}
{files and .jir executables}
\subsection{APIs for logger}

No documentation yet.

\subsubsection{\lstinline[basicstyle=\Large\ttfamily]$logger http connect$}

\apispec{cli: logger http connect | api: logger\_http\_connect | auth: admin}{host: str (*req), port: int (*req), url: str (*req), log: str (all)}
{Valid log params: {sys, app, all }}
\subsubsection{\lstinline[basicstyle=\Large\ttfamily]$logger http clear$}

\apispec{cli: logger http clear | api: logger\_http\_clear | auth: admin}{log: str (all)}
{Valid log params: {sys, app, all }}
\subsubsection{\lstinline[basicstyle=\Large\ttfamily]$logger list$}

\apispec{cli: logger list | api: logger\_list | auth: admin}{n/a}
{No documentation yet.}
\subsection{APIs for master}

\par
These APIs

\subsubsection{\lstinline[basicstyle=\Large\ttfamily]$master create$}

\apispec{cli: master create | api: master\_create | auth: user}{name: str (*req), global_init: str (), global_init_ctx: dict (\{\}), other_fields: dict (\{\})}
{other fields used for additional feilds for overloaded interfaces
(i.e., Dango interface)}
\subsubsection{\lstinline[basicstyle=\Large\ttfamily]$master get$}

\apispec{cli: master get | api: master\_get | auth: user}{name: str (*req), mode: str (default), detailed: bool (False)}
{Valid modes: {default, }}
\subsubsection{\lstinline[basicstyle=\Large\ttfamily]$master list$}

\apispec{cli: master list | api: master\_list | auth: user}{detailed: bool (False)}
{No documentation yet.}
\subsubsection{\lstinline[basicstyle=\Large\ttfamily]$master active set$}

\apispec{cli: master active set | api: master\_active\_set | auth: user}{name: str (*req)}
{NOTE: Specail handler included in general interface to api}
\subsubsection{\lstinline[basicstyle=\Large\ttfamily]$master active unset$}

\apispec{cli: master active unset | api: master\_active\_unset | auth: user}{n/a}
{No documentation yet.}
\subsubsection{\lstinline[basicstyle=\Large\ttfamily]$master active get$}

\apispec{cli: master active get | api: master\_active\_get | auth: user}{detailed: bool (False)}
{No documentation yet.}
\subsubsection{\lstinline[basicstyle=\Large\ttfamily]$master self$}

\apispec{cli: master self | api: master\_self | auth: user}{detailed: bool (False)}
{No documentation yet.}
\subsubsection{\lstinline[basicstyle=\Large\ttfamily]$master delete$}

\apispec{cli: master delete | api: master\_delete | auth: user}{name: str (*req)}
{No documentation yet.}
\subsection{APIs for object}

\par
...

\subsubsection{\lstinline[basicstyle=\Large\ttfamily]$global get$}

\apispec{cli: global get | api: global\_get | auth: user}{name: str (*req)}
{No documentation yet.}
\subsubsection{\lstinline[basicstyle=\Large\ttfamily]$object get$}

\apispec{cli: object get | api: object\_get | auth: user}{obj: element (*req), depth: int (0), detailed: bool (False)}
{No documentation yet.}
\subsubsection{\lstinline[basicstyle=\Large\ttfamily]$object perms get$}

\apispec{cli: object perms get | api: object\_perms\_get | auth: user}{obj: element (*req)}
{No documentation yet.}
\subsubsection{\lstinline[basicstyle=\Large\ttfamily]$object perms set$}

\apispec{cli: object perms set | api: object\_perms\_set | auth: user}{obj: element (*req), mode: str (*req)}
{No documentation yet.}
\subsubsection{\lstinline[basicstyle=\Large\ttfamily]$object perms default$}

\apispec{cli: object perms default | api: object\_perms\_default | auth: user}{mode: str (*req)}
{No documentation yet.}
\subsubsection{\lstinline[basicstyle=\Large\ttfamily]$object perms grant$}

\apispec{cli: object perms grant | api: object\_perms\_grant | auth: user}{obj: element (*req), mast: element (*req), read_only: bool (False)}
{No documentation yet.}
\subsubsection{\lstinline[basicstyle=\Large\ttfamily]$object perms revoke$}

\apispec{cli: object perms revoke | api: object\_perms\_revoke | auth: user}{obj: element (*req), mast: element (*req)}
{No documentation yet.}
\subsection{APIs for sentinel}

\par
A sentinel is a unit in Jaseci that represents the organization and management of
a collection of architypes and walkers. In a sense, you can think of a sentinel
as a complete Jac implementation of a program or API application. Though its the
case that many sentinels can be interchangeably across any set of graphs, most
use cases will typically be a single sentinel shared by all users and managed by an
admin(s), or each users maintaining a single sentinel customized for their
individual needs. Many novel usage models are possible, but I'd point the beginner
to the model most analogous to typical server side software development to start
with. This model would be to have a single admin account responsible for updating
a single sentinel that all users would share for their individual graphs. This
model is achieved through using \texttt{sentinel\_register},
\texttt{sentinel\_active\_global}, and \texttt{global\_sentinel\_set}.

\subsubsection{\lstinline[basicstyle=\Large\ttfamily]$sentinel register$}

\apispec{cli: sentinel register | api: sentinel\_register | auth: user}{name: str (default), code: str (), code_dir: str (./), mode: str (default), encoded: bool (False), auto_run: str (init), auto_run_ctx: dict (\{\}), auto_create_graph: bool (True), set_active: bool (True)}
{Auto run is the walker to execute on register (assumes active graph
is selected)}
\subsubsection{\lstinline[basicstyle=\Large\ttfamily]$sentinel pull$}

\apispec{cli: sentinel pull | api: sentinel\_pull | auth: user}{set_active: bool (True), on_demand: bool (True)}
{No documentation yet.}
\subsubsection{\lstinline[basicstyle=\Large\ttfamily]$sentinel get$}

\apispec{cli: sentinel get | api: sentinel\_get | auth: user}{snt: sentinel (None), mode: str (default), detailed: bool (False)}
{Valid modes: {default, code, ir, }}
\subsubsection{\lstinline[basicstyle=\Large\ttfamily]$sentinel set$}

\apispec{cli: sentinel set | api: sentinel\_set | auth: user}{code: str (*req), code_dir: str (./), encoded: bool (False), snt: sentinel (None), mode: str (default)}
{Valid modes: {code, ir, }}
\subsubsection{\lstinline[basicstyle=\Large\ttfamily]$sentinel list$}

\apispec{cli: sentinel list | api: sentinel\_list | auth: user}{detailed: bool (False)}
{No documentation yet.}
\subsubsection{\lstinline[basicstyle=\Large\ttfamily]$sentinel test$}

\apispec{cli: sentinel test | api: sentinel\_test | auth: user}{snt: sentinel (None), detailed: bool (False)}
{No documentation yet.}
\subsubsection{\lstinline[basicstyle=\Large\ttfamily]$sentinel active set$}

\apispec{cli: sentinel active set | api: sentinel\_active\_set | auth: user}{snt: sentinel (*req)}
{No documentation yet.}
\subsubsection{\lstinline[basicstyle=\Large\ttfamily]$sentinel active unset$}

\apispec{cli: sentinel active unset | api: sentinel\_active\_unset | auth: user}{n/a}
{No documentation yet.}
\subsubsection{\lstinline[basicstyle=\Large\ttfamily]$sentinel active global$}

\apispec{cli: sentinel active global | api: sentinel\_active\_global | auth: user}{auto_run: str (), auto_run_ctx: dict (\{\}), auto_create_graph: bool (False), detailed: bool (False)}
{Exclusive OR with pull strategy}
\subsubsection{\lstinline[basicstyle=\Large\ttfamily]$sentinel active get$}

\apispec{cli: sentinel active get | api: sentinel\_active\_get | auth: user}{detailed: bool (False)}
{No documentation yet.}
\subsubsection{\lstinline[basicstyle=\Large\ttfamily]$sentinel delete$}

\apispec{cli: sentinel delete | api: sentinel\_delete | auth: user}{snt: sentinel (*req)}
{No documentation yet.}
\subsection{APIs for stripe}

Set of APIs to expose jaseci stripe management

\subsubsection{\lstinline[basicstyle=\Large\ttfamily]$stripe product create$}

\apispec{cli: stripe product create | api: stripe\_product\_create | auth: admin}{name: str (VIP Plan), description: str (Plan description)}
{No documentation yet.}
\subsubsection{\lstinline[basicstyle=\Large\ttfamily]$stripe product price set$}

\apispec{cli: stripe product price set | api: stripe\_product\_price\_set | auth: admin}{productId: str (*req), amount: float (50), interval: str (month)}
{No documentation yet.}
\subsubsection{\lstinline[basicstyle=\Large\ttfamily]$stripe product list$}

\apispec{cli: stripe product list | api: stripe\_product\_list | auth: admin}{detalied: bool (True)}
{No documentation yet.}
\subsubsection{\lstinline[basicstyle=\Large\ttfamily]$stripe customer create$}

\apispec{cli: stripe customer create | api: stripe\_customer\_create | auth: admin}{paymentId: str (*req), name: str (cristopher evangelista), email: str (imurbatman12@gmail.com), description: str (Myca subscriber)}
{No documentation yet.}
\subsubsection{\lstinline[basicstyle=\Large\ttfamily]$stripe customer get$}

\apispec{cli: stripe customer get | api: stripe\_customer\_get | auth: admin}{customerId: str (*req)}
{No documentation yet.}
\subsubsection{\lstinline[basicstyle=\Large\ttfamily]$stripe customer payment add$}

\apispec{cli: stripe customer payment add | api: stripe\_customer\_payment\_add | auth: admin}{paymentMethodId: str (*req), customerId: str (*req)}
{No documentation yet.}
\subsubsection{\lstinline[basicstyle=\Large\ttfamily]$stripe customer payment delete$}

\apispec{cli: stripe customer payment delete | api: stripe\_customer\_payment\_delete | auth: admin}{paymentMethodId: str (*req)}
{No documentation yet.}
\subsubsection{\lstinline[basicstyle=\Large\ttfamily]$stripe customer payment get$}

\apispec{cli: stripe customer payment get | api: stripe\_customer\_payment\_get | auth: admin}{customerId: str (*req)}
{No documentation yet.}
\subsubsection{\lstinline[basicstyle=\Large\ttfamily]$stripe customer payment default$}

\apispec{cli: stripe customer payment default | api: stripe\_customer\_payment\_default | auth: admin}{customerId: str (*req), paymentMethodId: str (*req)}
{No documentation yet.}
\subsubsection{\lstinline[basicstyle=\Large\ttfamily]$stripe subscription create$}

\apispec{cli: stripe subscription create | api: stripe\_subscription\_create | auth: admin}{paymentId: str (*req), name: str (*req), email: str (*req), priceId: str (*req), customerId: str (*req)}
{TODO: name and email parameters not used!}
\subsubsection{\lstinline[basicstyle=\Large\ttfamily]$stripe subscription delete$}

\apispec{cli: stripe subscription delete | api: stripe\_subscription\_delete | auth: admin}{subscriptionId: str (*req)}
{No documentation yet.}
\subsubsection{\lstinline[basicstyle=\Large\ttfamily]$stripe subscription get$}

\apispec{cli: stripe subscription get | api: stripe\_subscription\_get | auth: admin}{customerId: str (*req)}
{No documentation yet.}
\subsubsection{\lstinline[basicstyle=\Large\ttfamily]$stripe invoices list$}

\apispec{cli: stripe invoices list | api: stripe\_invoices\_list | auth: admin}{customerId: str (*req), subscriptionId: str (*req), limit: int (10), lastItem: str ()}
{No documentation yet.}
\subsection{APIs for super}

No documentation yet.

\subsubsection{\lstinline[basicstyle=\Large\ttfamily]$master createsuper$}

\apispec{cli: master createsuper | api: master\_createsuper | auth: admin}{name: str (*req), global_init: str (), global_init_ctx: dict (\{\}), other_fields: dict (\{\})}
{other fields used for additional feilds for overloaded interfaces
(i.e., Dango interface)}
\subsubsection{\lstinline[basicstyle=\Large\ttfamily]$master allusers$}

\apispec{cli: master allusers | api: master\_allusers | auth: admin}{num: int (0), start_idx: int (0)}
{return and start idx specfies where to start
NOTE: Abstract interface to be overridden}
\subsubsection{\lstinline[basicstyle=\Large\ttfamily]$master become$}

\apispec{cli: master become | api: master\_become | auth: admin}{mast: master (*req)}
{No documentation yet.}
\subsubsection{\lstinline[basicstyle=\Large\ttfamily]$master unbecome$}

\apispec{cli: master unbecome | api: master\_unbecome | auth: admin}{n/a}
{No documentation yet.}
\subsection{APIs for user}

\par
These User APIs enable the creation and management of users on a Jaseci machine.
The creation of a user in this context is synonymous to the creation of a master
Jaseci object. These APIs are particularly useful when running a Jaseci server
or cluster in contrast to running JSCTL on the command line. Upon executing JSCTL
a dummy admin user (super\_master) is created and all state is dumped to a session
file, though any users created during a JSCTL session will indeed be created as
part of that session's state.

\subsubsection{\lstinline[basicstyle=\Large\ttfamily]$user create$}

\apispec{cli: user create | api: user\_create | auth: public}{name: str (*req), global_init: str (), global_init_ctx: dict (\{\}), other_fields: dict (\{\})}
{This API is used to create users and optionally set them up with a graph and
related initialization. In the context of
JSCTL, any name is sufficient and no additional information is required.
However, for Jaseci serving (whether it be the official Jaseci server, or a
custom overloaded server) additional fields are required and should be added
to the other fields parameter as per the specifics of the encapsulating server
requirements. In the case of the official Jaseci server, the name field must
be a valid email, and a password field must be passed through other fields.
A number of other optional parameters can also be passed through other feilds.
\vspace{3mm}\par
This single API call can also be used to fully set up and initialize a user
by leveraging the global init parameter. When set, this parameter attaches the
user to the global sentinel, creates a new graph for the user, sets it as the
active graph, then runs an initialization walker on the root node of this new
graph. The initialization walker is identified by the name assigned to
global init. The default empty string assigned to global init indicates this
global setup should not be run.\vspace{4mm}\par
\argspec{Parameters}{
\texttt{name} -- The user name to create. For Jaseci server this must be a valid
email address.\vspace{1.5mm}\par

\texttt{global\_init} -- The name of an initialization walker. When set the user is
linked to the global sentinel and the walker is run on a new active graph
created for the user.\vspace{1.5mm}\par

\texttt{global\_init\_ctx} -- Context to preload for the initialization walker\vspace{1.5mm}\par

\texttt{other\_fields} -- This parameter is used for additional fields required for
overloaded interfaces. This parameter is not used in JSCTL, but is used
by Jaseci server for the additional parameters of password, is\_activated,
and is\_superuser.\vspace{1.5mm}\par
}}
\subsection{APIs for walker}

\par
The walker set of APIs are used for execution and management of walkers. Walkers
are the primary entry points for running Jac programs. The
primary API used to run walkers is \textbf{walker\_run}. There are a number of
variations on this API that enable the invocation of walkers with various
semantics.

\subsubsection{\lstinline[basicstyle=\Large\ttfamily]$walker register$}

\apispec{cli: walker register | api: walker\_register | auth: user}{snt: sentinel (None), code: str (), encoded: bool (False)}
{Though the common case is to register entire sentinels, a user can also
register individual walkers one at a time. This API accepts code for a single
walker (i.e., \lstinline\{walker \{...\}\}).}
\subsubsection{\lstinline[basicstyle=\Large\ttfamily]$walker get$}

\apispec{cli: walker get | api: walker\_get | auth: user}{wlk: walker (*req), mode: str (default), detailed: bool (False)}
{Valid modes: {default, code, ir, keys, }}
\subsubsection{\lstinline[basicstyle=\Large\ttfamily]$walker set$}

\apispec{cli: walker set | api: walker\_set | auth: user}{wlk: walker (*req), code: str (*req), mode: str (default)}
{Valid modes: {code, ir, }}
\subsubsection{\lstinline[basicstyle=\Large\ttfamily]$walker list$}

\apispec{cli: walker list | api: walker\_list | auth: user}{snt: sentinel (None), detailed: bool (False)}
{No documentation yet.}
\subsubsection{\lstinline[basicstyle=\Large\ttfamily]$walker delete$}

\apispec{cli: walker delete | api: walker\_delete | auth: user}{wlk: walker (*req), snt: sentinel (None)}
{No documentation yet.}
\subsubsection{\lstinline[basicstyle=\Large\ttfamily]$walker spawn create$}

\apispec{cli: walker spawn create | api: walker\_spawn\_create | auth: user}{name: str (*req), snt: sentinel (None)}
{No documentation yet.}
\subsubsection{\lstinline[basicstyle=\Large\ttfamily]$walker spawn list$}

\apispec{cli: walker spawn list | api: walker\_spawn\_list | auth: user}{detailed: bool (False)}
{No documentation yet.}
\subsubsection{\lstinline[basicstyle=\Large\ttfamily]$walker spawn delete$}

\apispec{cli: walker spawn delete | api: walker\_spawn\_delete | auth: user}{name: str (*req)}
{No documentation yet.}
\subsubsection{\lstinline[basicstyle=\Large\ttfamily]$walker spawn clear$}

\apispec{cli: walker spawn clear | api: walker\_spawn\_clear | auth: user}{n/a}
{No documentation yet.}
\subsubsection{\lstinline[basicstyle=\Large\ttfamily]$walker yield list$}

\apispec{cli: walker yield list | api: walker\_yield\_list | auth: user}{detailed: bool (False)}
{No documentation yet.}
\subsubsection{\lstinline[basicstyle=\Large\ttfamily]$walker yield delete$}

\apispec{cli: walker yield delete | api: walker\_yield\_delete | auth: user}{name: str (*req)}
{No documentation yet.}
\subsubsection{\lstinline[basicstyle=\Large\ttfamily]$walker yield clear$}

\apispec{cli: walker yield clear | api: walker\_yield\_clear | auth: user}{n/a}
{No documentation yet.}
\subsubsection{\lstinline[basicstyle=\Large\ttfamily]$walker prime$}

\apispec{cli: walker prime | api: walker\_prime | auth: user}{wlk: walker (*req), nd: node (None), ctx: dict (\{\}), _req_ctx: dict (\{\})}
{No documentation yet.}
\subsubsection{\lstinline[basicstyle=\Large\ttfamily]$walker execute$}

\apispec{cli: walker execute | api: walker\_execute | auth: user}{wlk: walker (*req), prime: node (None), ctx: dict (\{\}), _req_ctx: dict (\{\}), profiling: bool (False)}
{No documentation yet.}
\subsubsection{\lstinline[basicstyle=\Large\ttfamily]$walker run$}

\apispec{cli: walker run | api: walker\_run | auth: user}{name: str (*req), nd: node (None), ctx: dict (\{\}), _req_ctx: dict (\{\}), snt: sentinel (None), profiling: bool (False), is_async: bool (False)}
{reports results, and cleans up walker instance.}
\subsubsection{\lstinline[basicstyle=\Large\ttfamily]$wapi$}

\apispec{cli: wapi | api: wapi | auth: user}{name: str (*req), nd: node (None), ctx: dict (\{\}), _req_ctx: dict (\{\}), snt: sentinel (None), profiling: bool (False)}
{No documentation yet.}
\subsubsection{\lstinline[basicstyle=\Large\ttfamily]$walker summon$}

\apispec{cli: walker summon | api: walker\_summon | auth: public}{key: str (*req), wlk: walker (*req), nd: node (*req), ctx: dict (\{\}), _req_ctx: dict (\{\}), global_sync: bool (True)}
{along with the walker id and node id}
\subsubsection{\lstinline[basicstyle=\Large\ttfamily]$walker callback$}

\apispec{cli: walker callback | api: walker\_callback | auth: public}{nd: node (*req), wlk: walker (*req), key: str (*req), ctx: dict (\{\}), _req_ctx: dict (\{\}), global_sync: bool (True)}
{along with the walker id and node id}
\subsubsection{\lstinline[basicstyle=\Large\ttfamily]$walker queue$}

\apispec{cli: walker queue | api: walker\_queue | auth: user}{task_id: str ()}
{No documentation yet.}



